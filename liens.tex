\section{Les liens}
Les liens sont  définis sous Unix. Ils permettent
d'obtenir une vue d'un fichier ou d'un répertoire \og comme s'il était
là\fg. Un lien utilise très peu de ressources.

Sur ma machine, la commande

\com{ls -l /}

donne (extrait):

\begin{verbatim}
lrwxrwxrwx   1 root root     7 avril 27  2020 lib -> usr/lib
lrwxrwxrwx   1 root root     9 avril 27  2020 lib32 -> usr/lib32
lrwxrwxrwx   1 root root     9 avril 27  2020 lib64 -> usr/lib64
lrwxrwxrwx   1 root root    10 avril 27  2020 libx32 -> usr/libx32
\end{verbatim}
le \ttt{l} en tête indique qu'on a affaire à un lien; par exemple,
\ttt{lib} est un \emph{lien symbolique} vers le répertoire 
  \ttt{usr/lib}.

  On peut vérifier ensuite que les commandes  \com{ls /lib} et
  \com{ls /usr/lib} donnent le même résultat.

  Un lien symbolique:
  \begin{itemize}
  \item S'installe avec la commande \com{ln -s}. Exemple:

    \com{ln -s /usr/lib lienVersLib}

    \ttt{lienVersLib} pointe alors vers \ttt{/usr/lib}.
  \item Peut pointer vers n'importe quoi: fichier ou répertoire.
  \item S'efface avec la commande \ttt{rm}. Exemple:

    \com{rm lienVersLib}.
  \item  \textdbend Effacer le lien est sans problème (ici
    \ttt{lienVersLib}), mais si on efface l'objet pointé, le lien persiste,
    mais ne pointe plus sur rien! Exemple:

    \com{touch toto}

    \com{ln -s toto tutu} (\ttt{tutu}  est un lien vers \ttt{toto}).

    \com{rm toto} (\ttt{tutu} ne pointe plus sur rien).

    \com{less tutu}
    \begin{center}
      \ttt{tutu: Aucun fichier ou dossier de ce type}
    \end{center}
    
  \end{itemize}

  Les liens symboliques sont assez agréables; ils permettent de créer des
  sortes d'alias, et puis de contourner un peu la rigidité de
  l'aborescence de fichiers: un lien peu pratiquement pointer de
  n'importe quel endroit vers un autre.
      
Il existe d'autres sortes de liens, les liens \og hard\fg, assez peu
intéressants en pratique pour les utilisateurs:\label{hardlink}
\begin{itemize}
\item Ils ne peuvent pointer que vers des fichiers.
\item Ils équipent le fichier pointé d'un compteur de référence;
  ainsi:

  \com{ln toto tutu} (tutu est un lien \og hard\fg{} vers \ttt{toto}
  (notez l'absence de l'option \ttt{-s}). La commande \ttt{ls} montre
    2 fichiers, alors que les données n'existent qu'une seule fois.

  \com{rm toto} 

  \com{less tutu} : tout se passe bien.
\end{itemize}
  
  
