\section{Forme (presque générale) des commandes Unix}
  \`A titre d'exmple, quelques unes des  différentes formes que
 peut prendre la commande \ttt{ls}:

  \begin{enumerate}
  \item \com{ls}
  \item \com{ls /tmp }

  \item   \com{ls /tmp /usr/ /bin}

    (ou \com{ls /var/log/*.log}, on
    verra ça  plus loin).%page \pageref{regular}).

  \item \com{ls -l /tmp /usr/ /bin}
  \item \com{ls -l -t -r /tmp}

    mais en pratique on tapera plutôt:

    \com{ls -ltr /tmp}
    
  \end{enumerate}

  On voit la structure d'une commande (les crochets \ttt{\{  \}}
  indiquent quelque chose de facultatif): 

  \begin{center}
    \texttt{commande \{-options\} \{-options\} \{arguments\}}
  \end{center}

  Une description plus exacte de la syntaxe nécessiterait d'utiliser
  la forme de Backus--Naur (voir
  \url{https://fr.wikipedia.org/wiki/Forme_de_Backus-Naur} par
  exemple), ce qui n'est pas vraiment prévu dans ce cours.

  On remarque qu'il existe en général une forme simple, sans
  arguments (1), une forme avec des arguments (2)(3) et qu'on modifie le
  comportement de la commande (on le complexifie) en ajoutant une ou
  plusieurs options, le ``-'' précédant chaque option ou chaque groupe
  d'options.

  Dans certains cas les options peuvent avoir elles aussi des
  arguments. Par exemple, la commande utilisée pour imprimer un fichier:

  \begin{itemize}
    \item \com{lpr chemin\_vers\_le\_fichier}

      imprimera le fichier sur l'imprimante standard.

    \item \com{lpr -P imp2 chemin\_vers\_le\_fichier}

      imprimera le fichier sur l'imprimante de nom \ttt{imp2}.
  \end{itemize}
  
  

  
