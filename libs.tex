\section{Bibliothèques (libraries)}

Une bibliothèque est un ensemble  de pièces de programmes qui peuvent
être réutilisées dans un programme. L'intérêt est qu'elles fournissent
des fonctions, des structures qui peuvent être utilisées par
différents programmes (voir la figure \ref{bib} et, par exemple, la
référence  \cite{wplib}).

Exemple: la bibliothèque \ttt{libc} (plus exactement \ttt{glibc})
fournit un grand nombre de fonctions plutôt basiques; une fonction
comme \ttt{time} qui permet de récupérer dans un programme l'heure,
telle qu'elle est connue par la machine. Un comprend donc l'intérêt a
avoir une seule version de cette méthode.

\subsection{Bibliothèques statiques et bibliothèques dynamiques}
\begin{itemize}
  \item Avec les bibliothèques statiques, les fonctions dont un
    programme a besoin sont prises dans les bibliothèques au moment de
    la construction du programme (édition des liens). Le fichier du
    programme (exemple  \ttt{/usr/bin/firefox}) contient à peu près
    tout ce qu'il faut pour faire tourner le programme.

  \item avec les bibliothèques dynamiques, on va charger en mémoire
    des fonctions dont on a besoin au moment de leur utilisation.
\end{itemize}

\subsubsection*{Quelles sont les bibliothèques dynamiques utilisées
  par un programme?}

La commande est \ttt{ldd}. Exemples:

\begin{itemize}
\item \com{ldd /usr/bin/firefox}. Réponse:

  \ttt{n'est pas un exécutable dynamique}. Donc on utilise des
  bibliothèques statiques.

\item \com{ldd /bin/ls}
  
\begin{verbatim}
  	linux-vdso.so.1 (0x00007fff9af8b000)
	libtinfo.so.6 => /lib/x86_64-linux-gnu/libtinfo.so.6 (0x00007ff135523000)
	libdl.so.2 => /lib/x86_64-linux-gnu/libdl.so.2 (0x00007ff13551d000)
	libc.so.6 => /lib/x86_64-linux-gnu/libc.so.6 (0x00007ff135333000)
	/lib64/ld-linux-x86-64.so.2 (0x00007ff1356b9000)
moi@kepler:/2/home/moi/CoursLinux/Cours/Cours4$ man ldd
moi@kepler:/2/home/moi/CoursLinux/Cours/Cours4$ ldd /bin/ls
	linux-vdso.so.1 (0x00007ffd3351f000)
	libselinux.so.1 => /lib/x86_64-linux-gnu/libselinux.so.1 (0x00007fdeb6035000)
	libc.so.6 => /lib/x86_64-linux-gnu/libc.so.6 (0x00007fdeb5e4b000)
	libpcre2-8.so.0 => /lib/x86_64-linux-gnu/libpcre2-8.so.0 (0x00007fdeb5dbb000)
	libdl.so.2 => /lib/x86_64-linux-gnu/libdl.so.2 (0x00007fdeb5db5000)
	/lib64/ld-linux-x86-64.so.2 (0x00007fdeb60c0000)
	libpthread.so.0 => /lib/x86_64-linux-gnu/libpthread.so.0 (0x00007fdeb5d93000)

\end{verbatim}
\
On voit entre autres, que l'on utilise la \ttt{libc}
(\ttt{libc.so.6}), ce qui est normal, car cette bibliothèque contient
beaucoup d'utilités de base.
\end{itemize}

Les intérêts des bibliothèques sont nombreux: ne pas redévelopper ce
qui existe, et utiliser des ``services'' partagés entre le plus grand
nombre de programmes améliore la fiabilité de ces services.

On va voir d'autres bibliothèques quand on va étudier les interfaces graphiques.
\begin{figure}
  \input libstikz
  \caption{Programme et bibliothèque}
  \label{bib}
\end{figure}\medskip

\begin{boxedminipage}{0.95\textwidth}\label{solve}
\begin{minipage}{0.48\textwidth}
  Un autre exemple: résoudre le système d'équations linéaires ci-contre.

  On peut évidemment coder ça \textsl{à la main},  mais le mieux est
  d'utiliser une bibliothèque libre (en pratique: \ttt{lapack}): plus
  de 30 ans 
  d'expérience, fiabilité et performances
  imbattables

  Intéressé(e) par cet exemple? voir en appendice
    page \pageref{lu}. 
\end{minipage}
\begin{minipage}{0.48\textwidth}
\begin{alignat*}{4}
   2x & {}+{} &  y & {}+{} & 3z & {}={} & 10 \\
    x & {}+{} &  y & {}+{} &  z & {}={} &  6 \\
    x & {}+{} & 3y & {}+{} & 2z & {}={} & 13
\end{alignat*}
\end{minipage}
\end{boxedminipage}

\subsubsection*{Nommage des bibliothèques dynamiques}
Pour la \ttt{libc} qui est un bon exemple, on voit:

\ttt{/usr/lib32/libc.so.6}

\begin{itemize}
\item L'extension \ttt{so} est celle des bibliothèques dynamiques.
\item Le  \ttt{.6} est un numéro de version.
\end{itemize}
