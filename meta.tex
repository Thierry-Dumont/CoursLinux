\section{Noms de fichiers, méta-caractères, et expressions régulières}
Un exemple (à tester) pour commencer:

\com{ls /var/log}

\noindent on voit qu'il y a  plein de fichiers dont le nom se termine par
\ttt{log} (par exemple \ttt{boot.log}). Comment ne lister que ceux ci?

Pour cela, on utilise le méta-caractère
``*'':

\com{ls /var/log/*.log}

et on ne voit qu'eux.

\underline{Explication}: ``*'' correspond à n'importe quelle chaîne de caractères
(formée de caractères alphabétiques, numériques et autres).
La commande \com{ls /var/log/*.log} va donc lister tous les fichiers
de \ttt{/var/log} dont le nom commence par n'importe quelle chaîne de
caractère, mais finit par \ttt{.log}.

Quelques autres exemples d'utilisation du jocker ``*'':
\begin{itemize}
\item On peut lister
tous les fichiers dont le nom commence par exemple par \ttt{syslog}:

\com{ls /var/log/syslog*}

\item On voit que certains fichiers dont le nom commence par
  \ttt{syslog} sont compressés (extension
  \ttt{.gz}).

  Pour ne lister qu'eux, je peux faire:

\com{ls /var/log/syslog*gz} ou dans notre cas \com{ls /var/log/syslog.*.gz}.
\end{itemize}\smallskip


\begin{center}
  \begin{minipage}{0.75\linewidth}
    \begin{itemize}
    \item \emph{Le méta-caractère ``*'' correspond à n'importe quelle
  chaîne de caractères, de longueur quelconque.} 
    \item \emph{Le méta-caractère ``?'', lui, correspond à un
        seul caractère, quelconque.}
    \end{itemize}
\end{minipage}
\end{center}
\underline{Exemple} (tester):


\com{ls /var/log/syslog.?.gz}

ou bien

\com{ls /var/log/syslog???gz}

dans ce cas, les 3 caractères ``matchés'' par \ttt{???} peuvent être
chacun quelconques et différents.

\paragraph{D'autres manières de filtrer:}
\begin{itemize}
\item Supposons que je veuille filtrer sur un seul caractère, numérique. Je
définis alors un intervalle de recherche \ttt{[0-9]}, puis:

\com{ls /var/log/syslog.[0-9].gz}

\item Si je veux filtrer sur les noms qui commencent par \ttt{a},
  \ttt{b} ou \ttt{c}, je peux 
faire:

\com{ls /var/log/[a-c]*}

\item Évidemment on peut tout mixer:

\com{ls /var/log/[a-c]*.[0-9].gz}

et aussi être plus restrictif:

\com{ls /var/log/[a-c]*.[1-3].gz}
\end{itemize}

Attention: l'ordre des caractères alphabétiques est
\texttt{ABC....Zabc...z} et par 
conséquent \ttt{[a-B]} est un intervalle vide. En revanche \ttt{[A-z]}
est l'ensemble de tous les caractères alphabétiques\footnote{C'est
  vrai tant qu'on se cantonne à l'utilisation des caractères ASCII;
  aujourd'hui on peut représenter \og tous\fg{} les caractères (Unicode).}. 


  On peut évidemment utiliser cette technique avec à peu près toutes les
  commandes.
  Exemples:

  \com{rm *.log}

  \com{mv toto.* /tmp}\footnote{\ttt{mv}: déplace, mais aussi renomme.}

  \begin{center}
    \begin{boxedminipage}{0.75\linewidth}
   Ceci n'est qu'une infime introduction à un monde qu'il serait bon de
  connaître pour traiter des chaînes de caractères: \emph{les
    expressions régulières}.
    \end{boxedminipage}
  \end{center}


  Mais comment est-ce que ça marche? Il faut en dire un peu plus sur
  le mécanisme du shell. Que se passe-t-il quand on tape une commande?
  Réponse: la commande est interprétée par le shell:

  \begin{enumerate}
    \item Ce qu'on tape est une chaîne de caractères (se terminant par
      le caractère ``retour chariot'').
    \item Le shell (bash ou autre) analyse cette chaîne. Il peut y
      avoir des erreurs lexicographiques (emploi de caractères pas
      reconnus) ou syntaxiques.

      Essayez par exemple:

      \com{\&ls}
    \item Le shell extrait le premier mot de la ligne, considérant que
      ce mot se termine au premier caractère ``espace'', et considère
      que c'est une commande. Il cherche alors cette
      commande\footnote{On va voir plus loin où il cherche.}; s'il ne
      la trouve pas, il vous le dit, sinon il lance la commande en
      lui passant le reste de la ligne comme \emph{argument}. Exemple:
      
      \com{ls -l /tmp}

      la 
      commande \ttt{ls} reçoit ``\ttt{-l /tmp}''\footnote{Notons que les
        espaces en trop sont supprimés: plusieurs espaces successifs se
        comportent comme un  seul.}.
    \item Mais si la ligne de commande contient une expression régulière,
      celle-ci est évaluée et la chaîne résultante est  passée à la
      commande: par  exemple,  si dans mon répertoire 
      courant il existe les 
      fichiers \ttt{toto.txt} et \ttt{toto.pdf}, la commande

      \com{ls -l toto.*}

      est évaluée en:

      \com{ls -l toto.pdf toto.txt}

      et
      c'est la chaîne de caractères ``\ttt {-l toto.pdf toto.txt}'' qui
      est passée à \ttt{ls}.
    \item Là s'arrête le travail du shell proprement dit: il passe la
      main à la commande  \ttt{ls}, créant ainsi un
        processus dont il est le père, et dont il garde le contrôle; \ttt{ls}
      analyse cette ligne: 
      est elle \emph{bien formée?} (la syntaxe des options est-elle
      correcte, etc.): le processus \ttt{ls} va faire ce qu'il a à faire, et
      finalement renvoyer au shell un message ``tout s'est bien passé'',
      ou bien un message d'erreur, et le shell reprend
      complètement la main, et attend de vous une autre commande.


      Derrière tout ça il y a la notion importante,
      mais hors de l'horizon de ce cours, de \emph{grammaire formelle}
      et d'\emph{analyse syntaxique}. Le shell est un interpréteur de
      commande sophistiqué qui permet d'écrire des programmes dans un
      langage bien défini, qui suis une \emph{grammaire} bien précise.
   \end{enumerate} 
