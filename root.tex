\section{Deux manières d'acquérir des pouvoirs exceptionnels}
Comme on l'a vu, les utilisateurs \og standard\fg{} n'ont pas tous les
pouvoirs\footnote{Entre autres, ils n'ont pas accès à certains
  fichiers ou répertoires, ou bien des accès restreints (lecture seule
  par exemple).}, mais il est nécessaire ne serait-ce que pour administrer la
machine (ce qui inclut par exemple les mises à jour) d'acquérir des
pouvoirs exceptionnels.

Les distributions Linux actuelles proposent deux méthodes:

\begin{itemize}
\item \emph{Le superutilisateur:}
C'est un utilisateur de  nom  \textsf{root}. Il a, par définition, tous les
droits. Donc, il est dangereux d'être ``root''\footnote{ne jamais être
  root en état d'ivresse.}.
\item \emph{La commande \ttt{sudo}:}
  elle permet d'exécuter une commande en étant root, le temps que dure
  la commande: 

  \com{sudo rm quelque-chose}

  par exemple.
\end{itemize}

L'idée, avec \ttt{sudo} est de ne donner que temporairement les
privilèges de root, en contrôlant le plus possible ce qui est fait, car:
\begin{itemize}
  \item Il faut avoir le droit de faire ``sudo''; pour cela il faut
    être membre du \emph{groupe sudo:}

    \com{grep sudo /etc/group}

    pour voir ça.

  \item Il faut donner son propre mot de passe.

  \item Le mot de passe vous donne le bien droit de faire ``sudo'', mais ce
    droit est limité dans le temps
\end{itemize}

Regardez ce que donnent:

\com{whoami}

et

\com{sudo whoami}
    

Évidemment, la sécurité de ``sudo'' est assez illusoire.

L'utilisateur ``root `` est toujours présent: les
utilisateurs d'Ubuntu --par exemple-- pourront le vérifier en regardant
le fichier  \ttt{/etc/shadow}\footnote{\com{sudo grep root
    /etc/shadow}, bien sûr!} où un
``!'' désactive la possibilité de se connecter.

En fait, c'est assez dépendant des distributions Linux. Sous les
debian, sudo n'est pas installé par défaut, et root est
un utilisateur comme les autres, avec un mot de passe. Sous Ubuntu on
installe sudo par défaut, mais il suffit de donner un 
mot de passe\footnote{Question: comment faire? La commande pour changer
  \emph{votre} mot de passe est \com{passwd}. Pour changer ou définir
  celui de \ttt{root}, il faut... être \ttt{root}.} à l'utilisateur root pour pouvoir aussi se connecter en tant
que root (depuis l'écran de connexion par exemple).

Notez enfin que \com{sudo -i}, c'est exactement comme s'être connecté en
tant que root.

Quelques avantages de sudo:
\begin{itemize}
\item On contrôle (via le groupe sudo) qui peut acquérir les
  privilèges de  ``root''.
\item Il faut déjà être connecté pour pouvoir faire sudo.
\item On peut configurer assez finement ce qui est autorisé pour les
  utilisateurs qui peuvent faire sudo (voir les fichiers et
  répertoires dont le nom commence par sudo dans \ttt{/etc}).
\end{itemize}
