\section{Comment explorer tout ça sous Linux?}\label{netlinux}
\subsection{La configuration de ma machine}
Dans \ttt{/etc}, il y a un certian nombre de fichiers et de
répertoires dont le nom commance par \ttt{network}. avce la
configuration automatique par DHCP, il n'y a plus trop de raisons
d'aller les regarder.

Le fichier \ttt{/etc/services} est intéressant, même si il n'y a pas
vraiment de raisons de le modifier: ce fichier contient, pour \og
tous\fg{} les protocoles réseau, le port utilisé (\emph{well known
  port}) et le protocole de transport utilisé (\ttt{tcp} ou
\ttt{udp}). Voici par exemple les lignes correspondant à \ttt{http} et
  \ttt{https}: 
\begin{verbatim}
http	80/tcp	www		# WorldWideWeb HTTP
https	443/tcp			# http protocol over TLS/SSL
\end{verbatim}

\subsubsection{La commande \ttt{ip}}

\com{ip address show} ou simplement \com{ip a}

va montrer quelque chose qui ressemble à ceci:

{\small
\begin{verbatim}
1: lo: <LOOPBACK,UP,LOWER_UP> mtu 65536 qdisc noqueue state UNKNOWN group default qlen 1000
    link/loopback 00:00:00:00:00:00 brd 00:00:00:00:00:00
    inet 127.0.0.1/8 scope host lo
       valid_lft forever preferred_lft forever
    inet6 ::1/128 scope host 
       valid_lft forever preferred_lft forever
2: enp3s0: <BROADCAST,MULTICAST,UP,LOWER_UP> mtu 1500 qdisc fq_codel state UP group default qlen 1000
    link/ether d8:50:e6:ba:2e:d8 brd ff:ff:ff:ff:ff:ff
    inet 192.168.0.10/24 brd 192.168.0.255 scope global dynamic noprefixroute enp3s0
       valid_lft 30777sec preferred_lft 30777sec
    inet6 2a01:e34:ec0e:46d0::4ab7:9bd8/128 scope global dynamic noprefixroute 
       valid_lft 63106sec preferred_lft 63106sec
\end{verbatim}        
}

Pour une description détaillée de tous ces résultats, voir par exemple
\cite{comip}.  De façon résumée, \ttt{lo} et \ttt{enp3s0} siont deux
interfaces réseau:
\begin{enumerate}
  \item \ttt{lo} (loopback) est une interface qui n'est pas
    \emph{physique}. Elle permet à la machine de se connecter à
    elle-même, par exemple.
  \item \ttt{enp3s0}: ici, une carte ethernet (cela pourrait être une
    interface wi-fi). Détaillons un peu:
    \begin{itemize}
      \item \ttt{link/ether d8:50:e6:ba:2e:d8 brd ff:ff:ff:ff:ff:ff}:
        c'est l'adressse MAC.
      \item \ttt{inet 192.168.0.10/24}: c'est l'adressse IPv4 et le
        netmask (24).
      \item \ttt{inet6 fe80::b9fd:1b45:9712:eff9/64}: l'adressse IPv6
        et son netmask (64). Come le netmask est 64, il reste 64
        autres bits libres, ce qui est probablement l'ensemble
        d'adresses IPv6 disponibles pour \ttt{free.fr} ($2^{64} \simeq
        1.8 \times 10^{19}$ adressses!).
    \end{itemize}
\end{enumerate}

On peut se limiter aux renseignements concernant IPv6 (ou IPv6) en
utilisant respectivement les commandes:

\com{ip -4 a} et \com{ip -6 a}
\subsubsection{ping}
Utilisant ICMP (\emph{Internet Control Message Protocol}), le
protocole de message de contrôle sur Internet),  \ttt{ping} envoie une
demande d'echo à unemachnide du réseau. Exemples:

\com{ping berkeley.edu}
\begin{verbatim}
ING berkeley.edu (35.163.72.93) 56(84) bytes of data.
64 bytes from ec2-35-163-72-93.us-west-2.compute.amazonaws.com (35.163.72.93): icmp\_seq=1 ttl=29 time=167 ms
64 bytes from ec2-35-163-72-93.us-west-2.compute.amazonaws.com (35.163.72.93): icmp\_seq=2 ttl=29 time=167 ms
\end{verbatim}
(Control-C pour interrompre. Mais la machine distante est libre de ne
pas répondre! (ça se configure).

\com{ping localhost} utilisera l'interface \ttt{lo} de votre machine
(tester).
\subsubsection{tracepath (ou traceroute)} Permet de savoir par où on
passe pour atteindre une adresse donnée. Tenir compte du fait que
certaines machines sont à l'interface de deux réseaux et ont plusieurs
adresses.

\com{tracepath aldil.org}

(à tester pour s'amuser).

\subsection{whois}
Utile pour savoir qui gère un domaine. Exemple:

\com{whois thierry-dumont.fr}

\subsubsection{nslookup}
Questionner un DNS:


\com{nslookup aldil.org}

mais aussi:

\com{nslookup 80.67.185.24}

\subsubsection{dig}

\subsubsection{nmap}
