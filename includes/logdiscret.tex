
\section{Une autre \og source\fg{} de méthodes pour la cryptographie:
  le logarithme discret}

On peut partir de quelque chose de simple (mais rien n'est difficile
ici): une horloge qui n'indique que les heures, numérotées 0, 1, 2,
3,...,10,11.

On sait bien que s'il est 10 heures, dans 4 heures il sera 2
heures. Plus exactement:
\begin{itemize}
  \item l'addition est définie ainsi: si $a$ et $b$ sont des \og
    heures\fg, le résultat de $a+b$ est le reste de la division de
    $a+b$ par $12$. ainsi $10 + 4 = 14$ et le reste de $14$ divisé par
    $12$ est $2$. Donc, dans notre arithmétique de l'horloge, $10+4 =
    2$.
  \item On peut de la même manière définir la multiplication: si $a$
    et $b$ sont deux nombres de l'intervalle 0,..,11 (bornes
    incluses), alors on définit $a \times b$ comme le reste de la
    division du produit de $a$ par $b$ dividé par $12$. Exemples:
    \begin{itemize}
      \item $5 \times 7$ est le reste de $35$ divisé par $12$, soit
        $11$.
      \item $4 \times 3 =0$.
    \end{itemize}
\end{itemize}

On appelle ces nombres les \emph{entiers modulo 12}.

Cela devient intéressant quand on remplace $12$ par un nombre
premier. Ainsi pour $7$, donc  pour les \emph{entiers modulo 7},  on obtient
les tables d'addition et de multiplication suivantes:


\begin{minipage}{0.5\textwidth}
  \begin{center}
\begin{tabular}{c|ccccccc}
+&0&1&2&3&4&5&6\\
\hline
0&0&1&2&3&4&5&6\\
1&1&2&3&4&5&6&0\\
2&2&3&4&5&6&0&1\\
3&3&4&5&6&0&1&2\\
4&4&5&6&0&1&2&3\\
5&5&6&0&1&2&3&4\\
6&6&0&1&2&3&4&5\\
\end{tabular}
\smallskip

Table d'addition.
\end{center}
\end{minipage}
\begin{minipage}{0.5\textwidth}
  \begin{center}
\begin{tabular}{c|ccccccc}
*&0&1&2&3&4&5&6\\
\hline
0&0&0&0&0&0&0&0\\
1&0&1&2&3&4&5&6\\
2&0&2&4&6&1&3&5\\
3&0&3&6&2&5&1&4\\
4&0&4&1&5&2&6&3\\
5&0&5&3&1&6&4&2\\
6&0&6&5&4&3&2&1\\
\end{tabular}
\smallskip

Table de multiplication.
\end{center}
\end{minipage}
\medskip

et en supprimant les $0$ pour la table de multiplication:

\begin{center}
\begin{tabular}{c|cccccc}
*&1&2&3&4&5&6\\
\hline
1&1&2&3&4&5&6\\
2&2&4&6&1&3&5\\
3&3&6&2&5&1&4\\
4&4&1&5&2&6&3\\
5&5&3&1&6&4&2\\
6&6&5&4&3&2&1\\
\end{tabular}
\end{center}

On peut facilement vérifier que:
\begin{itemize}
 \item Par exemple:
   $3^6 = 1$, $3^2=2$, $3^1 = 3$, $3^4= 4$, $3^5=5$ et $3^3= 6$
 \item Les nombres $3^p$ parcourent toutes les valeurs
   possibles (1, 2,...,6), et une seule fois. C'est vrai aussi pour
   $5$.
   Mais ce n'est pas vrai pour $4$, pour lequel les nombres $4^p$
   parcourent juste les valeurs $4$, $2$, et $1$.

   On dit que $3$ et $5$ sont des \emph{générateurs}. 
\end{itemize}



On choisit un générateur $a$. Le logarithme
discret\footnote{sous-entendu: dans les entiers modulo 7.}  à base $a$
de $x$ est 
le nombre $n$ tel que $a^n = x$, ce qu'on écrit $\log_a (x) = n$. Par
exemple, (voir ci-dessus) $3^3= 6$, donc $\log_3(6)= 3$ (sous-entendu:
dans les entiers modulo 7).

\begin{enumerate}
  \item Au lieu de calculer dans \emph{les entiers modulo 7} comme ci-dessus,
    on choisit (au hasard)  un très grand nombre premier $P$ (120
    chiffres, par exemple, 
    on a déjà vu que c'était facile),
    quelque chose comme
\begin{verbatim}
08499652057794414184160279146447576156019630125837339050723288132709
9544824336871492352319491041397008775809148491161129
\end{verbatim}
et on calcule dans  \emph{les entiers modulo ce nombre} (on ne va pas
écrire la table de multiplication!).
 \item Il existe de bons algorithmes pour trouver des
   \emph{générateurs}. On choisit un \emph{générateur} qu'on appelle $g$.
 \item On choisit un nombre $x$ compris entre $1$ et notre grand
   nombre premier $P$. C'est particulièrement facile. On calcule $k=
   g^x$. C'est facile.
 \item La clé \textbf{publique}, c'est $(g,k)$.
 \item Et la clé \textbf{privée} c'est $x$.

   On a $x = \log_g k$: $x$
   est le logarithme discret de base $g$ (sous entendu dans les
   entiers module $P$) de $k$. \textbf{Et ce problème (le calcul du
     logarithme discret) est un problème à coût non polynomial), donc
     beaucoup trop coûteux pour être effectué}.
\end{enumerate}

