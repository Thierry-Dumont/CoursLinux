\section{Comment explorer tout ça sous Linux?}\label{netlinux}
\subsection{La configuration de ma machine}
Dans \ttt{/etc}, il y a un certain nombre de fichiers et de
répertoires dont le nom commence par \ttt{network}. avec la
configuration automatique par DHCP, il n'y a plus trop de raisons
d'aller les regarder.

Le fichier \ttt{/etc/services} est intéressant, même si il n'y a pas
vraiment de raisons de le modifier: ce fichier contient, pour \og
tous\fg{} les protocoles réseau le port utilisé (\emph{well known
  port}) et le protocole de transport utilisé (\ttt{tcp} ou
\ttt{udp}). Voici par exemple les lignes correspondant à \ttt{http} et
\ttt{https}\footnote{\com{grep http /etc/services}}:\smallskip

\begin{verbatim}
http	80/tcp	www		# WorldWideWeb HTTP
https	443/tcp			# http protocol over TLS/SSL
\end{verbatim}

Donc \ttt{http} utilise le port 80 et  \ttt{https} le port 443; tous deux
utilisent \ttt{tcp}.

\subsubsection{La commande \ttt{ip}}La commande:

\com{ip address show}

ou simplement

\com{ip a}\footnote{La commande
  \ttt{ip} remplace et généralise la commande \ttt{ifconfig}.}

va montrer quelque chose qui ressemble à ceci:\smallskip

{\small
\begin{verbatim}
1: lo: <LOOPBACK,UP,LOWER_UP> mtu 65536 qdisc noqueue state UNKNOWN group default qlen 1000
    link/loopback 00:00:00:00:00:00 brd 00:00:00:00:00:00
    inet 127.0.0.1/8 scope host lo
       valid_lft forever preferred_lft forever
    inet6 ::1/128 scope host 
       valid_lft forever preferred_lft forever

2: enp3s0: <BROADCAST,MULTICAST,UP,LOWER_UP> mtu 1500 qdisc fq_codel state UP group default qlen 1000
    link/ether d8:50:e6:ba:2e:d8 brd ff:ff:ff:ff:ff:ff
    inet 192.168.0.10/24 brd 192.168.0.255 scope global dynamic noprefixroute enp3s0
       valid_lft 35117sec preferred_lft 35117sec
    inet6 2a01:e34:ec0e:46d0::4ab7:9bd8/128 scope global dynamic noprefixroute 
       valid_lft 81139sec preferred_lft 81139sec
    inet6 fe80::b9fd:1b45:9712:eff9/64 scope link noprefixroute 
       valid_lft forever preferred_lft forever

    
\end{verbatim}        
}\smallskip

Pour une description détaillée de tous ces résultats, voir par exemple
\cite{comip}.  De façon résumée, \ttt{lo} et \ttt{enp3s0} sont deux
interfaces réseau:
\begin{enumerate}
  \item \ttt{lo} (loopback) est une interface qui n'est pas
    \emph{physique}. Elle permet à la machine de se connecter à
    elle-même, de permettre à des programmes de dialoguer \og comme
    si\fg{} c'était à travers un vrai réseau.
  \item \ttt{enp3s0}: ici, une carte ethernet (cela pourrait être une
    interface wi-fi). Détaillons un peu:
    \begin{itemize}
      \item \ttt{192.168.0.10/24}: c'est l'adresse IPv4, avec le
        netmask de 24 bits, comme on l'a vu plus haut.
      \item \ttt{2a01:e34:ec0e:46d0::4ab7:9bd8/128}: c'est une adresse
        IPv6. On remarque que le netmask est de 128 bits, ce qui veut
        dire que cette adresse est seule dans son réseau. C'est
        l'adresse publique, celle qu'on voit quand on va sur le site
        \url{https://www.whatismyip.com/fr/}.
      
      \item \ttt{fe80::b9fd:1b45:9712:eff9/64}: une adresse IPv6
        et son netmask (64 bits de gauche). Comme le netmask est de 64
        bits,
        il reste $128-64  = 64$
        autres bits libres ($2^{64} \simeq
        1.8 \times 10^{19}$ adresses!). Cette adresse n'est pas
        publique (propriété générale des adresses commençant par
        \ttt{fe80}). On a ainsi  un réseau IPv6 local.

      \item L'interface physique est donc associée à 3 adresses: une
        adresse IPv4 et deux adresses IPv6.
    \end{itemize}
\end{enumerate}

On peut se limiter aux renseignements concernant IPv4 (ou IPv6) en
utilisant respectivement les commandes:

\com{ip -4 a}

et

\com{ip -6 a}
\subsubsection{ping}
Utilisant ICMP (\emph{Internet Control Message Protocol}), le
protocole de message de contrôle sur Internet),  \ttt{ping} envoie une
demande d'écho à une machine du réseau. Exemples:

\com{ping berkeley.edu}\smallskip

\begin{verbatim}
ING berkeley.edu (35.163.72.93) 56(84) bytes of data.
64 bytes from ec2-35-163-72-93.us-west-2.compute.amazonaws.com (35.163.72.93): icmp\_seq=1 ttl=29 time=167 ms
64 bytes from ec2-35-163-72-93.us-west-2.compute.amazonaws.com (35.163.72.93): icmp\_seq=2 ttl=29 time=167 ms
\end{verbatim}
(Control-C pour interrompre). Mais la machine distante est libre de ne
pas répondre! (ça se configure).\smallskip

\com{ping localhost} utilisera l'interface \ttt{lo} de votre machine
(tester).
\subsubsection{tracepath (ou traceroute)} Permet de savoir par où on
passe pour atteindre une adresse donnée. Tenir compte du fait que
certaines machines sont à l'interface de deux réseaux et ont plusieurs
adresses: on obtiendra donc en général plus d'adresses que de machines
physiques. 

\com{tracepath aldil.org}

(à tester pour s'amuser).

\subsection{whois}
Utile pour savoir qui gère un domaine. Exemple:

\com{whois thierry-dumont.fr}

\subsubsection{nslookup}
Questionner un DNS:


\com{nslookup aldil.org}

mais aussi:

\com{nslookup 80.67.185.24}

%\subsubsection{dig}

\subsubsection{nmap}
\texttt{nmap} sert à \emph{scanner} les ports d'une machine. Exemple:

\com{nmap aldil.org}

renvoie la liste des ports ouverts sur  \ttt{aldil.org}:

\begin{verbatim}
Not shown: 992 filtered ports
PORT    STATE SERVICE
22/tcp  open  ssh
25/tcp  open  smtp
80/tcp  open  http
143/tcp open  imap
443/tcp open  https
465/tcp open  smtps
587/tcp open  submission
993/tcp open  imaps
\end{verbatim}



\begin{itemize}
\item les ports 25, 143, 587 et 993 correspondent au serveur de
  courrier électronique,
\item les ports http et https sont ouverts pour le serveur web,
\item sur le port 22, ssh permet une connexion cryptée sur la machine,
\end{itemize}

bref, que du nécessaire\footnote{La machine de l'Aldil est une machine
  virtuelle; elle est directement sur l'internet, et pas derrière une
  quelconque box qui ferait office de filtre}.

\subsection{Ouvrir ou fermer des ports: le firewall} \label{firewall}
Un des intérets de 
la notion de port est qu'on peut filtrer intelligemment ceux qui sont
ouverts ou fermés. Le firewall est intégré à Linux:
\begin{itemize}
  \item \ttt{Netfilter}\cite{netfilter} est implanté au sein du noyau
    linux.
  \item \ttt{iptables}\cite{iptables} est un programme de l'espace
    utilisateur, pour configurer \ttt{netfilter}\footnote{Il
      faut être super-utilisateur pour l'utiliser, bien sûr.}.
\end{itemize}

On peut utiliser ces outils pour implanter un NAT (voir page
\pageref{nat}): donc vous ne devez pas être très étonné d'apprendre que
vos \og box\fg{} sont des systèmes Linux.

L'utilisation des  \ttt{iptables}, sans être difficile, est hors du
champ de ce cours.

On peut recommander aux utilisateurs qui ne veulent pas investir trop
dans les  \ttt{iptables} de se tourner vers des outils qui en
simplifient l'usage (et certainement en restreignent un les
possibilités) comme \ttt{ufw}\cite{ufw}, simple à utiliser. On
recommande toujours de procéder de la manière suivante, \emph{si c'est
possible:}
\begin{enumerate}
\item Fermer tous les ports.
\item Ouvrir ensuite seulement ceux dont on a besoin.
\end{enumerate}
