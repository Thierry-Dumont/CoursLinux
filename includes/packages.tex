
\section{Installation de packages fournis par la distribution}
Dans cette section on regarde le premier cas, avec les outils des
distributions \og Debian-like\fg{} (Debian, Ubuntu, Mint,...).

Il existe des outils graphiques comme \ttt{Synaptic}, bien agréables,
mais on s'interesse ici la ligne de commande.

\subsection{La commande apt}
C'est une version modernisée de la commande \ttt{apt-get}.

Rappelons que la distribution Debian a été la première à gérer
correctement les dépendances entre packages. De quoi s'agit t-il?
La commande \ttt{apt show} nous le dit; exemple:
\begin{center}
  \com{apt show tesseract-ocr}
\end{center}

\begin{verbatim}
Depends: libarchive13 (>= 3.2.1), libc6 (>= 2.29), libcairo2 (>=
1.2.4), libfontconfig1 (>= 2.12.6), libgcc-s1 (>= 3.0), libglib2.0-0
(>= 2.12.0), libicu67 (>= 67.1-1~), liblept5 (>= 1.75.3),
libpango-1.0-0 (>= 1.37.2), libpangocairo-1.0-0 (>= 1.22.0),
libpangoft2-1.0-0 (>= 1.14.0), libstdc++6 (>= 5.2), libtesseract4 (=
4.1.1-2build3), tesseract-ocr-eng (>= 4.00~), tesseract-ocr-osd (>=
4.00~) 
\end{verbatim}

Cela nous dit que le package \ttt{teseract-ocr} dépend (utilise) des
bibliothèques (les \ttt{lib*}), mais aussi d'autres programmes (\ttt{
  tesseract-ocr-eng} et \ttt{ tesseract-ocr-osd}). Et à
chaque fois, la commande nous dit quel numéro de version minimal doit
être utilisé.

Donc quand on va installer un package comme 
\ttt{teseract-ocr}, il faut que le système de gestion de packages
installe aussi tous les packages dont il dépend. De même en cas de mises à
jour, il faut les propager si nécessaire.

\subsubsection{Principales options de la commande \ttt{apt}}

\begin{itemize}
\item \ttt{apt search}. Plusieurs possibilités:
  \begin{itemize}
  \item \ttt{apt search tesseract-ocr} par exemple.
  \item \ttt{apt search ``tesseract*''} (tous les packages dont le nom
    commence par \ttt{tesseract}; on peut mettre n'importe quelle
    \emph{expression régulière}).
  \end{itemize}
\item \ttt{apt show} vu ci-dessus.
\item \ttt{apt install}: installer un package. Exemple:
  \begin{center}
  \com{apt install tesseract-ocr}
  \end{center}
  installe le package et ses dépendances (plus exactement: les
  dépendances d'abord, puis le package).
\item \ttt{apt remove} et \ttt{apt purge}: supprime un package. Avec
  l'option \ttt{purge}, la suppression est radicale: on efface tous
    les fichiers de configuration. Attention, on ne supprime pas les
    dépendances. Pour supprimer les dépendances, il faut utiliser la
    commande \ttt{apt autoremove}.
\end{itemize}
\subsubsection{Les mises à jour}
Il faut utiliser successivement 2 commandes:

\begin{enumerate}
\item \ttt{ apt update}: met à jour la liste des packages. On va
  pouvoir ensuite comparer les numéros de version disponibles avec ce
  qui est installé. Alors, la commande:
\item   \ttt{ apt upgrade}. Met à jour ce qui doit l'être.
\end{enumerate}

Pour la  mise à jour ou l'installation de packages: les packages sont
d'abord téléchargés (fichiers avec l'extension \ttt{.deb}), puis
installés. La suppression des  \ttt{.deb} n'est \emph{pas
  automatique}:
il convient de lancer la commande \ttt{apt clean} après une
installation ou une mise à jour.

Notons aussi que dans une mise à jour il y a non seulement
installation des nouvelles versions des packages, mais redémarrage des
services qui y sont associés, s'il en existe. Par exemple, le démon
\ttt{systemd} sera relancé après une mise à jour du programme (du
fichier) \ttt{systemd}.

\subsection{Faut-il rebooter la machine?}
Oui, dans certains cas, en particulier quand le noyau a été mis à jour
(package \ttt{linuximage}). Les interfaces graphiques l'indiquent, pas
celles en ligne de commande.
Après le redémarrage (car c'est seulement après le redémarrage qu'on
va utiliser le nouveau noyau), il convient de lancer la commande 

\begin{center}
\com{apt autoremove}
\end{center}

pour supprimer l'ancienne version du noyau.
