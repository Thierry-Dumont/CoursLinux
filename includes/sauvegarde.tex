\section{Une application complète: sauvegarde d'un répertoire avec
  rsync}



\subsection*{But}: réaliser une sauvegarde \og intelligente \fg{} d'un
répertoire vers un autre (à priori sur un support externe, connecté en
usb).

\subsubsection*{Rsync} La commande rsync est très
puissante\footnote{\ttt{rsync} n'est pas forcément installé par
  défaut:

  \com{sudo apt install rsync}

  ou l'équivalent pour l'installer.}; à priori
c'est une simple copie d'un répertoire vers un autre, par exemple:

\com{rsync -a /home/ /mnt/home/}

va copier le répertoire \ttt{/home/} vers \ttt{/mnt/home/}. Mais
attention, \emph{on ne va copier que ce qui n'est pas déjà présent dans}
\ttt{/mnt/home/}, \emph{ou ce qui est déjà présent dans} \ttt{/mnt/home/}
 {mais a été modifié dans}
\ttt{/home/}.  Le but est de synchroniser (d'où le nom de la
commande) les deux répertoires en  faisant le minimum de
travail. L'option \ttt{-a} est nécessaire (on conserve les droits, les
propriétaires, les groupes des objets). Attention, les chemins (ici
\ttt{/home/} et  \ttt{/mnt/home/} doivet avoir un \ttt{/} à la fin.

\begin{itemize}
\item On peut aussi ajouter \ttt{\tirr delete} (oui, il y a bien
  deux ``\ttt{-}''): 

  \com{rsync -a \tirr delete /home/ /mnt/home/}

  alors, si des fichiers existent dans \ttt{/mnt/home/} sans exister
  dans \ttt{/home/}, ils sont effacés dans \ttt{/mnt/home/}.
\end{itemize}
  
  \textbf{Avec ces deux options, on est donc capable de faire une
    sauvegarde parfaite\footnote{Le contenu de \ttt{/mnt/home/} sera
      bien identique à celui de \ttt{/home/} après.}.}
  
  Autres options de \ttt{rsync}, souvent utiles:

  \begin{itemize}
    \item Verbose: \texttt{-v}: vous dit tout ce que fait rsync. C'est
      rapidement beaucoup trop bavard, mais c'est très utile pour la
      mise au point!

    \item \texttt{\tirr dry-run}: ne transfère rien, mais, associé à
      \ttt{-v}  permet de voir ce qui se passerait sans l'option.
  \end{itemize}

  Avant toute copie, ou la première fois il est \large{vraiment}
  recommandé de faire un test à blanc:

  \com{rsync -av \tirr delete \tirr dry-run /home/ /mnt/home/}

  Bien regarder ce qui se passe!\footnote{Il peut être judicieux de
    faire: \com{rsync -av \tirr delete \tirr dry-run /home/ /mnt/home/|less}}
    
\subsection{La sauvegarde} On va d'abord faire les choses \og à la
main\fg{}, et puis on automatisera la sauvegarde.

Que faut il faire?

\begin{enumerate}
  \item Monter la partition sur laquelle on va faire la sauvegarde, si
    c'est nécessaire.
  \item Avec \ttt{rsync} synchroniser les deux répertoires: celui à
    sauvegarder et la sauvegarde.
  \item Éventuellement, démonter la partition de sauvegarde.
\end{enumerate}

Ne pas garder la partition de sauvegarde montée en permanence diminue
les risques d'écriture intempestifs\footnote{Et aussi, certains
  disques externes s'éteignent une fois démontés.}

Supposons que ma partition de sauvegarde soit toujours
\ttt{/dev/sdd1}, qu'on la monte sur \ttt{/mnt} et qu'on veuille
  sauvegarder \ttt{/home}. Alors, les trois étapes ci-dessus sont:
\begin{enumerate}
  \item \com{mount /mnt}\footnote{Là, je suppose qu'il y a la bonne ligne
    dans \ttt{/etc/fstab}.  Sinon, on peut toujours faire:

    \com{mount -t ext4 /dev/sdd1 /mnt}

    tout aussi efficace.
  }
    
  \item Synchroniser avec \ttt{rsync}.  La commande est:

    \com{rsync -a \tirr delete /home/ /mnt/home/}
    

    Options à considérer, en tout cas pour la mise au point:

     \begin{itemize}
     \item \ttt{-v}: \emph{verbose}, bavard.
     \item \ttt{\tirr dry-run}: ne rien faire, ne rien copier, ne rien
       effacer.
     \end{itemize}

     Donc, pour la mise au point, on peut vérifier que tout se passe
     bien avec:

     \com{rsync -av \tirr dry-run  \tirr delete /home/
       /mnt/home/}\footnote{Il n'y a 
       aucun problème à interrompre rsync avec Control+C}

     \textdbend\textdbend

     \textbf{Ne pas oublier les `''/'' \`A LA FIN pour les
       répertoires:  \ttt{/home/}, \ttt{/mnt/home/} (et non pas
       \ttt{/home} ni \ttt{/mnt/home}).} 

     \textdbend\textdbend
       
     
  \item Éventuellement, démonter la partition de sauvegarde:

    \com{umount /mnt}
\end{enumerate}

\subsection{Automatiser la sauvegarde}
Le but est de faire une sauvegarde qui se déclenche automatiquement
tous les jours.

\subsubsection{Première étape: faire un \og script\fg}
Un script, c'est un programme pour le shell.\label{save1}

Pour cela on va créer un fichier qui contient les commandes vues précédemment
(après avoir tout testé à la main!). Voici le fichier, qu'on peut
  appeler par exemple \ttt{sauv.sh}.

  \begin{center}
  \begin{boxedminipage}{0.85\textwidth}
\begin{verbatim}
#!/bin/bash
mount /mnt
rsync -a --delete /home/ /mnt/home/ &>>/var/log/sauv.log
umount /mnt
\end{verbatim}
  \end{boxedminipage}
  \end{center}

Commentaires:
\begin{itemize}
  \item La première ligne est là là pour dire que le script doit être exécuté
    avec le shell \emph{bash}.
  \item  \ttt{\&>>/var/log/sauv.log}. On se souvient de la
    redirection des entrées-sorties; En fait il y a plusieurs sorties
    numérotées 1 (normale), 2 (erreur). Le \ttt{\&} redirige toutes
    les sorties; on les redirige vers \ttt{/var/log/sauv.log} et les deux ``>``
    (\texttt{>>}) rallongent le fichier  \ttt{/var/log/sauv.log} (on
    n'écrase pas le contenu de  \ttt{/var/log/sauv.log}, ce qui serait
    le cas avec un seul \ttt{>}).
\end{itemize}

Où installer ce fichier? par exemple dans \ttt{/etc/cron.daily}. Une
fois créé avec l'éditeur de texte, on le rend exécutable (\com{chmod
  +x /etc/cron.daily/sauv.sh}).


Comment l'exécuter automatiquement ? C'est fait! Il tournera tous les
jours où la machine fonctionnera (c'est parce qu'on l'a mis dans
\ttt{/etc/cron.daily}).


Il reste une chose à faire: le fichier  \ttt{/var/log/sauv.log} va
grossir indéfiniment. Il faut le faire \og tourner\fg{} avec
\ttt{logrotate}.

On peut regarder les différents scripts présents dans
\ttt{/etc/logrotate.d}
et en déduire que celui ci devrait convenir (ce n'est pas très
critique; \com{man logrotate} en dira plus sur les options du fichier):
\begin{center}
  \begin{boxedminipage}{0.85\textwidth}
  \begin{verbatim}
/var/log/sauv.log
{
	rotate 4
	weekly
	missingok
	notifempty
	compress
	delaycompress
	sharedscripts
	endscript
}
\end{verbatim}
  \end{boxedminipage}
\end{center}

Installer ce fichier sous le nom \ttt{sauve} par exemple dans
\ttt{/etc/logrotate.d/}. 

\subsection{Sauvegarder plusieurs répertoires}
On peut assez facilement transformer le \emph{script} ci-dessus page
\pageref{save1} pour 
sauvegarder plusieurs répertoires. Il peut être intéressant de
sauvegarder \ttt{/home}, mais aussi \ttt{/etc} par exemple, qui
contient toute la configuration de la machine. Pour cela, on va un peu
programmer le shell (car le shell est un langage de programmation).

\subsubsection{Un premier script}
Fabriquons\footnote{Il faut utiliser un éditeur de texte, genre
  \ttt{nano} ou autre.} un fichier de nom \ttt{essai1.sh} (le nom n'a
pas d'importance):
\begin{center}
\begin{boxedminipage}{0.85\textwidth}
\begin{verbatim}
#!/bin/bash
for d in /home/ /etc/;do
echo $d
done
\end{verbatim}
\end{boxedminipage}
\end{center}
Nous avons déjà rencontré la première ligne. \`A la deuxième ligne nous
commencons une boucle; la variable \ttt{d}\footnote{On aurait pu
  choisir un nom complètement différent.} va contenir successivement
les chaînes de caractères \ttt{/home/} et \ttt{/etc/}. Tout de qui est
compris entre la ligne \ttt{for...; do} et la ligne \ttt{done} est
effectué successivement avec les valeurs \ttt{/home/} et \ttt{/etc/}
stockés dans la 
  variable \ttt{d}. \`A la
troisième ligne, on écrit le contenu de \ttt{d} (le shell fait une
différence entre une variable (\ttt{d} ici) et son contenu qui ici est
\ttt{\$d}). Enfin la dernière ligne termine la boucle \ttt{for}.
Une fois le fichier écrit, rendons le exécutable:

\com{chmod +x essai.sh}

et lançons la commande:

\com{./essai.sh}.
On voit s'imprimer successivement  \ttt{/home/} et \ttt{/etc}.
\subsubsection{Deuxième étape} On réécrit le script vu page
\pageref{save1}:

\begin{center}
  \begin{boxedminipage}{0.85\textwidth}
\begin{verbatim}
#!/bin/bash
mount /mnt
for d in /home/ /etc/;do
rsync -a --delete $d /mnt/$d &>>/var/log/sauv.log
done
umount /mnt
\end{verbatim}
  \end{boxedminipage}
  \end{center}
Et voilà: \ttt{d} contiendra succesivement \ttt{/home/} et \ttt{/etc/},
\ttt{/mnt/\$d} vaudra succesivement \ttt{/mnt/home/} et
\ttt{/mnt/etc/}, ce qui fera bien ce qu'on cherche\footnote{
  rsync crée automatiquement le répertoire de destination s'il
  n'existe pas.}. Notons que
\ttt{/etc/} ne subissant que peu de modifications chaque jour, il n'y
aura pratiquement aucun travail à effectuer pour le sauvegarder, \emph{après}
la première sauvegarde.

\subsection{Synchronisation à distance}
\ttt{Rsync} permet des synchronisations de répertoires entre machines
distantes, à condition qu'on puisse établir une connexion cryptée
(\ttt{ssh}) entre les machines.

Supposons que je dispose d'un compte utilisateur \ttt{untel}  sur la
machine  distante \ttt{mm.xxx.yy}. Je synchroniserai
le répertoire distant \ttt{/sauv/} (par exemple) avec le
répertoire \ttt{/home/} de ma machine locale avec la commande:

\com{rsync -a \tirr delete -e ssh /home/ untel@mm.xxx.yy:/sauv/}

il peut être bien utile d'ajouter l'option \ttt{-z} pour compresser
les données transférées et limiter la bande passante sur le réseau:

\com{rsync -az \tirr delete -e ssh /home/ untel@mm.xxx.yy:/sauv/}

Une telle sauvegarde est absolument sûre, car la liaison est
cryptée\footnote{Sur la machine \ttt{mm.xxx.yy}, on peut aussi faire
  en sorte que le répertoire \ttt{/sauv/} soit crypté, mais ce n'est
  pas du ressort de \ttt{rsync}.}

