\documentclass[10pt]{smfart}

\usepackage[T1]{fontenc}
\usepackage[french]{babel}
\usepackage[a4paper]{geometry}
\usepackage{graphicx,caption,subcaption}
\usepackage{xcolor,lmodern,footmisc,boxedminipage}
\usepackage{manfnt,enumitem,pifont,url}
\usepackage[linktocpage=true]{hyperref}
\usepackage{microtype}
\usepackage{appendix,tikz,csquotes}
\usetikzlibrary{shapes}
\usepackage[citestyle=numeric,bibstyle=numeric]{biblatex}
\addbibresource{a.bib}
%\usepackage{kpfonts}
\title{Unix au temps du couvre-feu (3)\\Compléments et exercices}
\author{Thierry Dumont \textsl{pour} l'ALDIL}
\date{\today}
%
\newcommand{\com}[1]{\texttt{cmd>#1}}
\newcommand{\comc}[2]{\texttt{>#1} \underline{#2}}
\newcommand{\exo}{\textbf{\underline{Exercice}:~}}
\newcommand{\exos}{\textbf{\underline{Exercices}:}}
\newcommand{\exx}[1]{\underline{\textbf{#1}}}
\newcommand{\ttt}[1]{\texttt{#1}}
\newcommand{\sff}{système de fichiers}
\newcommand{\sffs}{systèmes de fichiers}
\newcommand{\tirr}{\texttt{-{}-}}
%
\newcommand{\red}[1]{\textcolor{red}{#1}}
\newcommand{\blue}[1]{\textcolor{blue}{#1}}
\newcommand{\rmq}[1]{{\bfseries\underline{Remarques}:}\\#1}
\newcommand{\dx}[3]{
  \begin{boxedminipage}{0.4\linewidth}
    \begin{center}
      \textsf{Commande}
    \end{center}
    #1\end{boxedminipage}~$ \Rightarrow $
    \begin{boxedminipage}{0.4\linewidth}
    \begin{center}
      \textsf{Résultat} #3
    \end{center}
      #2\end{boxedminipage}
}
\setcounter{tocdepth}{1}
\begin{document}

\maketitle

\tableofcontents
\centerline{
  \begin{minipage}{0.65\textwidth}
    \textsl{Dans la suite la syntaxe:}

    \begin{center}
      \com{commande arguments}
    \end{center}

    \textsl{désignera
une commande (à taper dans le shell); le \og prompt\fg{}, c'est à dire
le début de la ligne de commande peut varier d'une configuration à l'autre;
} \com{} \textsl{représente ce prompt et n'est pas à taper.}
  \end{minipage}
}

\input includes/c11
\input includes/c21
% cours3
\input includes/comlinux
\input includes/meta
\input includes/root
\input includes/shell
\input includes/caches
\input includes/commandes
% cours 4
\input includes/libs
\input includes/interfacegra
\input includes/liens
\input includes/filesystem
\input includes/sauvegarde
\input includes/classic
\input includes/nettools
\input includes/prepackages
\input includes/packages
\input includes/source

\input includes/intross
\input includes/algocrypt
\input includes/certif
\appendix
  \appendixpage


  \input includes/lu
   \input includes/cow
  \input includes/btree

\printbibliography

\begin{flushright}
  \tiny Document composé en \LaTeX.
\end{flushright}
\end{document}
