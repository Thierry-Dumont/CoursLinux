\section{Interface graphique, bureau etc.}
Comment fonctionnent les systèmes de bureau, d'interfaces graphiques
que nous utilisons? C'est assez complexe, ne serait-ce que parce qu'on
hérite d'un passé assez lointain (les années 80). La complexité de ces
outils outrepasse celle du noyau Linux. Et puis depuis les débuts de
ces développements l'environnement matériel a beaucoup changé: les GPU
n'implantaient pas les fonctionalités actuelles. 


Une remarque préalable: vous pouvez utiliser une machine Linux sans
interface graphique. Tapez \ttt{CNTRL+Alt F1} et les plus anciens se
retrouveront face à quelque chose qui leur rappellera le DOS.

Linux offre plusieurs environnements de bureau permettant tous de
gérer des fenêtres des menus, des éléments graphiques etc. Citons:
Gnome, Xfce, Mate, Ukui, KDE, mais il en existe d'autres.

Schématiquement, un système de bureau comporte au moins 3 couches,
chacune incarnée par une ou plusieurs bibliothèques:

\begin{enumerate}
\item Un niveau \textsl{bas} qui sait créer des fenêtres, interagir
  avec le clavier et la souris du coté de l'utilisateur et, du coté
  machine avec des dispositifs gérant directement
  l'affichage. actuellement ce
  niveau peut être incarné par deux logiciels:

  \begin{enumerate}
  \item X11, \textsl{The X-Window system}.
  \item Wayland.
  \end{enumerate}

\item Un niveau intermédiaire, capable de gérer des boutons, des menus
  déroulants, le couper-coller, etc. C'est le plus souvent la
  bibliothèque GTK, développée à l'origine pour le logiciel de
  manipulation d'images GIMP.

\item Le niveau supérieur: c'est gnome, Xfce etc, etc.  mais en fait
  c'est un peu plus compliqué: Xfce par exemple (et certainement les
  autres aussi s'interfacent avec X11 et GTK.
\end{enumerate}

\subsection{X11 ou Wayland?}
X11 est ancien (le développement a commencé en 1984), lourd et réputé
plus lent que des outils équivalents sous Windows et MacOs. D'où
l'idée de développer quelque chose de plus léger et de plus
moderne. Il y a eu plusieurs propositions et il semble que Wayland
l'emporte.

Certaines distributions utilisent maintenant Wayland (Debian),
d'autres garde X11 (Ubuntu). Il faut savoir que pas mal d'applications
s'interfacent directement avec X11 et que, pour les faire fonctionner
sous Wayland, il a fallu développer... un serveur X11 sous Wayland, au
moins pour une période transitoire.
 X11 permet aussi l'affichage déporté (un programme tourne sur une
 machine, l'affichage est déporté sur une autre): l'intéret pour cette
 fonctionnalité diminue: on a longtemps utilisé des terminaux X,
 désormais obsolètes, et un outil comme x2go permet un bien meilleur
 affichage à distance (il est basé sur les techniques de compression
 d'images utilisées pour la télévision numérique).

 Regardons un peu le cas Xfce (les autres sont à coup sûr peu
 différents):

 La commande:

 \com{ps aux|grep -i xfce}

 liste plusieurs processus. C'est donc que:
 \begin{itemize}
 \item Xfce n'est pas monolithique.
 \item Il va obligatoirement que ces processus communiquent entre eux.
 \end{itemize}


 Parmi ces processus, on peut regarder 
 \ttt{xfce4-session} qui est sûrement important. La commande

 \com{ldd /usr/bin/xfce4-session}

 liste 82 bibliothèques! Explicitons en quelques unes:

 \begin{itemize}
 \item \ttt{libX11}: c'est bien sûr le fameux X11. Notons aussi
   d'autres bibliothèques qui ensemblent forment X11: libXrender,
   libXext etc. (tout ce dont le nom commence par libX).
 \item \ttt{libgtk}: définit principalement des \og widgets\fg{}
   (boutons, menus etc.).
 \item \ttt{libgdk}: \og en dessous \fg{} de \ttt{gtk}. C'est
   l'intermédiaire entre \ttt{gtk} et le système de fenêtrage (X11 ou
   Wayland ou...).
 \item \ttt{libpango}: gestion des polices de caractères.
 \item \ttt{libcairo}: dessins bi-dimensionnels, vectoriels. Utilise
   l'accélération graphique, si c'est possible.
 \item \ttt{libwayland}: la bibliothèque est présente bien que la
   machine utilise x11.
 \item \ttt{libdbus}: c'est une bibliothèque de communication. On en
   parle ci-dessous.
 \end{itemize}
\subsection*{Dbus}
Dbus est un \emph{bus logiciel}. Citons Wikipédia: \og \emph{D-Bus permet à
des programmes clients de s'enregistrer auprès de lui, afin d'offrir
leurs services aux autres programmes. Il leur permet également de
savoir quels services sont disponibles. Les programmes peuvent aussi
s'enregistrer afin d'être informés d'événements signalés (parce que
gérés) par le noyau, comme le branchement d'un nouveau
périphérique}\fg{}
(voir figure \ref{dbus}).

\begin{figure}
  \begin{center}
  \begin{tikzpicture}[thick,scale=0.6, every node/.style={scale=0.6}]
\node (A) [cylinder, shape border rotate=0, draw,minimum height=12cm,minimum width=1cm] {Dbus};
\draw[<->] (-4,0.5) -- (-4,2);
\draw (-5,2) rectangle (-3,4)   node[midway]{Processus};
\draw[<->] (0,0.5) -- (0,2);
\draw (-1,2) rectangle (1,4)   node[midway]{Processus};
\draw (2.5,-3.5) circle(1.5) node {Noyau};
\draw[->] (2.5,-2) -- (2.5,-0.5);
  \end{tikzpicture}
  \end{center}
  \caption{Dbus}
  \label{dbus}
\end{figure}

Il faut mentionner le projet \textsf{freedesktop.org}\cite{fd} qui
vise à l'interopérabilité des différents environnements de bureau, en
définissant des standards et en développant des bibliothèques
logicielles  (Cairo, par exemple).
