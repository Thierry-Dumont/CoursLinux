\section{Problématique}
 On va supposer que deux
personnes veulent dialoguer; il est habituel de les appeler Alice et
Bob.

Pour pouvoir communiquer de manière sécurisée il faut:
\begin{enumerate}
  \item Crypter les messages.
  \item S'assurer que, quand Alice écrit à Bob (et vice-versa), c'est
    bien Bob son 
    interlocuteur et non pas un usurpateur.
\end{enumerate}

\subsection{Comment crypter les messages?}
Il faut:
\begin{enumerate}
\item Définir un système de cryptographie qu'on puisse considérer
  comme  indéchiffrable.
\item Utiliser ce système de façon sûre.
  
\item Il faut aussi que crypter et décrypter ne soit pas trop
  \emph{coûteux} car on veut pouvoir échanger de grands messages (des
  fichiers de grande taille, par exemple).
\end{enumerate}

Les systèmes de cryptographie sont tous basés sur deux
  composants:
  \begin{enumerate}
  \item Un algorithme.
  \item Une ou plusieurs \og clés\fg{} qui paramètrent cet algorithme.
  \end{enumerate}
  
Les systèmes de cryptographie ont longtemps eu un problème au niveau
de la clé de chiffrage/déchiffrage: 
les systèmes classiques sont basés sur une seule clé qui sert à la
fois au chiffrage et au déchiffrage; on parle alors de cryptographie
symétrique. Si Alice et Bob veulent 
discuter, cela    nécessite qu'ils  partagent la même
clé. La partage de la clé est resté une faiblesse des meilleurs systèmes, comme
la machine Enigma allemande pendant la deuxième guerre mondiale: la
capture d'un sous-marin avec une valise de clés (les sous-marins
partaient pour longtemps et, par prudence, il fallait changer de clé
tous les jours) 
a grandement fait avancer la décryptage d'Enigma par les spécialistes
de Bletchley-Park. En temps de paix, jusqu'aux années 80, des
personnels spécialisés parcouraient le monde pour transporter les
clés, prenant moulte précautions. On verra qu'il est possible de
partager une clé de manière sûre.

\subsection{La cryptographie à clés asymétriques}
Ça a été le bond en avant de la cryptographie. 
\begin{enumerate}
\item L'algorithme utilisé pour crypter les messages par Bob et Alice
  utilise non pas une mais \emph{deux} clés différentes:  une pour
  chiffrer un message et 
   l'autre pour le déchiffrer.
\item Une des deux clés ne peut pas être déduite de l'autre.
\end{enumerate}

Ainsi Bob possède  deux clés:
\begin{itemize}
  \item Une est la clé \emph{privée}, qu'il garde précieusement
    cachée. C'est elle qui va servir à déchiffrer les messages.
  \item L'autre est \emph{publique}, qu'il peut diffuser autant qu'il
    veut sans précaution particulière.
\end{itemize}
La clé  \emph{privée est celle qui ne peut pas se déduire de la
  clé publique.} 

Supposons qu'Alice veuille écrire à Bob; elle le lui fait savoir par
un message envoyé \og en clair \fg. Alors:
\begin{itemize}
\item Bob envoie sa clé \emph{publique}
  à Alice;
\item Alice se sert ce 
  cette clé pour chiffrer le message envoyé à Bob.
\item  Bob se sert de sa clé
  \emph{privée} pour déchiffrer le message d'Alice.
\end{itemize}



Pour que ça marche, c'est à dire pour que ça garantisse la
confidentialité du message envoyé par Alice, il faut absolument qu'on ne puisse
pas \emph{déduire} la clé \emph{privée} de Bob de sa clé
\emph{publique} (qu'il a donnée à Alice). Mais alors, la clé publique
peut être diffusée autant qu'on veut, sans risque: elle ne sert qu'à
crypter, et il faut la clé privée pour décrypter.

  Bien sûr, dans
l'autre sens, Alice
renvoie sa clé  \emph{publique} à Bob, qui s'en sert pour lui écrire, etc.

\subsection{Le problème de la cryptographie asymétrique}
On verra plus loin comment on réalise  des systèmes cryptographiques
asymétriques; mais avant,  il faut parler de leur principal défaut:
ils sont trop 
coûteux pour traiter de grandes quantités d'information.

On contourne ce problème en n'utilisant le système à clés
asymétriques que pour échanger secrètement LA clé d'un système de
cryptographie symétrique: on établi une connexion avec un système de
chiffrage asymétrique: ce système est utilisé seulement pour l'échange
de LA clé d'un système de cryptage \emph{symétrique.}
Remarquons que le problème du partage de la clé, qui était le point
faible de tous les systèmes de cryptographie symétrique est
résolu\footnote{Mais en utilisant une technologie plus sophistiquée:
  le cryptage asymétrique!}. 

On limite en général la
\og durée de vie\fg{} de cette clé et, si nécessaire, on fabrique au
bout d'un temps fixé une
nouvelle clé pour de chiffrage \emph{symétrique} et on l'échange avec
le système de cryptage asymétrique.

Il faut \og seulement\fg{} que le système de cryptage
  \emph{symétrique} soit assez robuste pour ne pas être décrypté par
  quelqu'un qui ne possède pas la clé.

%% Voilà une métaphore intéressante:
%% \begin{itemize}
%%   \item Bob veut écrire secrètement à Alice.
%%   \item Bob et Alice ont chacun une boîte dont ils gardent jalousement
%%     la clé. Ainsi, si la boîte de Bob (d'Alice) est fermée, seul Bob
%%     (Alice) peut l'ouvrir.
%%   \item Bob envoit sa boîte \emph{ouverte} à Alice.
%%   \item Alice reçoit la boîte de Bob et met sa boîte ouverte dans celle de
%%     Bob.
%%   \item Alice ferme la boîte de Bob et lui envoit.
%%   \item Bob a reçu sa boite; il l'ouvre --ce qu'il est seul à pouvoir
%%     faire--; il découvre la boîte d'Alice, 
%%     ouverte. Il met son message dans la boîte d'Alice, la ferme et
%%     l'envoit à Alice. Alice est seule capable d'ouvrir sa boîte, et
%%     donc seule capable d'accéder au message de Bob.
%% \end{itemize}

\section{Un exemple: ssh (Secure Shell)}
\ttt{ssh} est à la fois un ensemble de programmes et un protocole (une
norme) pour la communication cryptée. Dans sa version la plus
populaire il permet une d'ouvrir un shell sur une machine lointaine:

\begin{center}
\com{ssh monLogin@machinelointaine.domaine}
\end{center}

Cette commande:
\begin{enumerate}
\item Ouvre une connexion cryptée asymétrique: le mot de passe est
  donc crypté.
\item Ensuite, on passe à une connexion cryptée, symétrique.
\end{enumerate}

D'autre commandes en dérivent, par exemple:

\begin{itemize}
\item \ttt{scp}: copie cryptée entre machines.
\item \ttt{sftp}: comme \ttt{ftp}, mais crypté.
\end{itemize}

On peut se servir de \ttt{ssh} d'un grand nombre de manières: par
exemple, avec \ttt{rsync} pour synchroniser des répertoires entre deux
machines distantes (pratique pour faire des backups très
sûrs). L'implantation de \ttt{ssh} qui tourne sous Linux est \ttt{openssh}.

Sur le serveur (la machine à laquelle on se connecte), doit tourner un
démon \ttt{sshd}. Lors de l'installation de ce logiciel, une paire de
clés (privée, publique) propre à la machine  est engendrée\footnote{Ce
  qui implique un tirage 
  de nombres au hasard.}.


\subsection{Quelques détails à propos de openssh}
\subsubsection{Coté serveur} C'est donc le démon \ttt{sshd} qui
tourne.
Sa configuration est dans \ttt{/etc/ssh}:
\begin{itemize}
  \item Le fichier \ttt{sshd\_config} contient la configuration. Les
    valeurs par défaut sont en général suffisantes, mais on peut par
    exemple, changer le port (22 par défaut) et obliger l'utilisation
    d'une connexion par clé (voir plus loin) ou interdire le login
    \ttt{root}.
  \item Plusieurs fichiers  dont le nom commence par \ttt{ssh\_host}
    contiennent les clés privées et publiques (suffixe \ttt{.pub}) du
    serveur.
    Pourquoi plusieurs? Parce que ssh peut utiliser plusieurs
    protocoles de cryptographie. C'est quelque chose qui se négocie
    lors de la connexion du client.

    \textbf{Important:} en cas de réinstallation du système, il est
    recommandé de sauvegarder ces clés et de les réinstaller sur le
    nouveau système. Sinon, on souffre quelques désagréments.
\end{itemize}
\subsubsection{Coté client} La configuration est dans le répertoire
caché \ttt{.ssh/}, dans le home directory. Elle est créée lors de la
première utilisation. On y trouve un fichier \ttt{config}.
  A priori, il n'y a pas grand chose à y mettre, il
  doit être vide par défaut.

 
