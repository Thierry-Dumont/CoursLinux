


\section{Installer depuis la \og source \fg}
On peut avoir besoin d'installer un programme ou une bibliothèque qui
n'est pas disponible comme package pré-compilé dans une distribution
(ou aussi si on veut une version plus récente que celle disponible).

Il y a deux cas un peu différents:
\begin{enumerate}
  \item Le cas d'un programme écrit dans un langage compilé (C,
    C++,...).
  \item Le cas d'un programme écrit dans un langage interprété, le cas
    emblématique étant Python.
\end{enumerate}

\emph{On regarde ici le premier cas.}

Les codes source sont en général stockés dans un répertoire; ce
répertoire contient en principe:
\begin{itemize}
  \item Des fragments de code (fichiers d'extensions: .h, .c, .cc
    etc.).
  \item Des procédures pour compiler (traduire) le code en langage
    machine et pour l'installer dans l'arborescence de la machine.
  \item Éventuellement de la documentation.
  \item Des modes d'emploi pour l'installation (fichiers README,
    INSTALL etc.)
\end{itemize}

Dans une première étape, on récupère le répertoire, qui
contient le code source, ensuite on \emph{compile}, puis on installe
le programme. Commençons par le processus de compilation et d'installation.

\subsection{Le processus de compilation}
C'est un processus complexe, mais on va l'automatiser.
\subsubsection{La compilation proprement dite}
Cela consiste à traduire chaque fragment de code (admettons que c'est
écrit en langage C, mais peu importe\footnote{Rappelons que C est
  \emph{le} langage de base des systèmes Unix: il a été créé pour
  développer Unix.}) en langage machine: à partir
du fichier, disons \ttt{toto.c}, on crée le fichier
\ttt{toto.o}. C'est un programme, le \emph{compilateur}, qui fait ça. La
commande \textsl{à la main} serait:

\com{gcc toto.c -o toto.o}\footnote{\ttt{gcc} est le compilateur
  standard sous Linux; sous les systèmes \emph{à la Debian}, il peut être
  installé par la commande \ttt{apt install build-essential}. Mais il
  existe d'autres compilateurs du langage C dans les distributions
  Linux, comme le compilateur \ttt{clang}.}

La commande \ttt{gcc} est en général accompagnée d'options pour mieux
optimiser le code engendré, l'adapter précisément au type de cpu de
la machine \footnote{Ce qui va probablement améliorer les
  performances, mais rendre très probablement le code binaire non
  portable d'une machine à l'autre.}, ou à une syntaxe particulière.
Un exemple typique de ligne de compilation pourra être:

\com{gcc -O 3 -m cpu=native toto.c -o toto.o}\footnote{Ce qui veut
  dire: optimiser au niveau 3 (fortement), et engendrer du code pour
  le cpu local.}
\subsubsection{L'édition des liens}
Une fois tous les fragments de programme traduits (compilés), il faut
les relier entre eux, leur associer des liens vers des bibliothèques
existantes, etc. C'est un travail assez complexe, fait par la commande
\ttt{ld} (que l'utilisateur, même s'il développe des programme, n'a pas
en général à connaître). Le résultat est un fichier, qui contient du
langage machine, prêt à être exécuté\footnote{à la différence des
  fichiers comme \ttt{toto.o} à qui il manque beaucoup de choses,
  comme les appels de bibliothèque, pour simplifier.}. Rappelons que l'exécution de ce programme consistera
entre autres, à  recopier ce fichier en mémoire.
\subsubsection{Installer le programme}
  Si tout s'est bien passé, il faut ensuite installer le programme
  créé (il peut y en avoir plusieurs!) dans un endroit accessible
  (dans le PATH des utilisateurs), c'est à dire copier un ou plusieurs
  fichiers dans un répertoire (dont le nom se termine par \ttt{bin} en
  général), et leur donner les bons droits (\ttt{x} pour tout le
  monde).

  On peut bien sûr, ensuite, se débarrasser du répertoire source.
  \subsubsection{Un processus trop complexe, et les remèdes}
Il faut imaginer que le code source peut être éclaté en des centaines,
des milliers ou plus de fichiers. Un certain nombre de techniques vont
simplifier et automatiser son installation.

\begin{enumerate}
\item La commande \ttt{make} (voir \cite{make}).

  Elle lit un fichier de nom \ttt{Makefile} (par exemple) et fait... ce
  qu'il faut, ou plutôt ce qu'on lui dit de faire.

  Exemple de fichier \ttt{Makefile}:
\begin{verbatim}
toto.o toto.c
    gcc toto.c -o toto.o
tutu.o tutu.c machin.h
    gcc tutu.c -o tutu.o
prog: toto.o tutu.o
    ld ...
\end{verbatim}
Ce qui se lit ainsi (commencer par le bas):
\begin{itemize}
\item Pour créer le programme \ttt{prog}, il faut auparavant créer
  \ttt{toto.o} et \ttt{tutu.o} et effectuer la commande \ttt{ld} qu'on
  ne détaille pas ici.
  \item Pour obtenir le fichier binaire \ttt{toto.o}, il faut compiler
    \ttt{toto.c} (deux premières lignes, la deuxième ligne indique ce
    qu'il faut faire pour cela). Idem pour obtenir
    \ttt{tutu.o}, il faut compiler \ttt{tutu.c}; ici on a une
    seconde dépendance, à partir de \ttt{machin.h} .
\end{itemize}
Avantage de cette technique: si le développeur change disons
\ttt{toto.c}, on n'a pas à recompiler \ttt{tutu.c}, mais seulement
\ttt{toto.c}  (mais il faut
recréer \ttt{prog} à partir des différents fichiers \ttt{*.o} (deux
dernières lignes du fichier \ttt{Makefile})). C'est
très important: certains projets comportent des milliers de fichiers
source!
\item Bien. Mais la création du fichier \ttt{Makefile} s'avère
  rapidement cauchemardesque. Pour automatiser son écriture, on a créé
  un ensemble de programmes, les \ttt{autotools} (voir
  \cite{autotools}). C'est fort complexe, mais il n'est pas nécessaire
  de les maîtriser si on n'est pas développeur\footnote{Des
    alternatives existent, plus simple comme \ttt{cmake} (voir
    \cite{cmake}).
     \ttt{cmake} est développé par la société kitware, spécialiste de
     visualisation scientifique, présente à Villeurbanne.}.

  Les   \ttt{autotools} engendrent un fichier \ttt{Makefile}, à partir
  de fichiers de configuration, réputés être simples (ça se discute),
  et permettent aussi de tenir compte de particularités de la machine
  et du système (suis-je sous Linux? sous Windows ? etc.).

  
\subsubsection{Concrètement, pour l'utilisateur}
Pour compiler / installer un programme, l'utilisateur doit seulement:
\begin{enumerate}
  \item Lancer un ou plusieurs des \ttt{autotools}; en général, il
    suffit de lancer:

    \com{./configure}

     dans le répertoire source (C'est en général documenté (fichier
     INSTALL ou/et README) dans le répertoire source).

    cette commande va engendrer le fichier \ttt{Makefile} \emph{si
    c'est possible}. Mais peut-être que des bibliothèques, des
      programmes vont manquer: le script \ttt{./configure} va vous le
      dire; il faudra les installer pour aller plus loin (avec un peu de
      chance, ce sont des packages de la distribution qui doivent être
      installés), puis relancer
      \ttt{./configure} (recommencer éventuellement jusqu'à ce que
      toutes les dépendances nécessaires soient satisfaites).
  \item Taper \ttt{make}\footnote{On peut essayer \ttt{make -j n} avec
    pour \ttt{n} le nombre de c{\oe}urs, mais ça ne fonctionne pas
    toujours!}

    et attendre que la compilation soit terminée.

    \item Si tout s'est bien passé, il faut lancer la commande
      \com{make install} ou plutôt \com{sudo make install}.

\end{enumerate}
\end{enumerate}
%Auparavant, il faut bien récupérer le répertoire source.

\textbf{Remarque:} sous Ubuntu, il faut au moins installer les paquets
\ttt{build-essential} et \ttt{autotools-dev} pour pouvoir compiler un
programme. 

\subsection{Récupérer le répertoire source}

Deux possibilités:
\begin{enumerate}
\item Fichier \ttt{tar}, en général compressé.
\item \og Repository\fg{} \ttt{git}.
\end{enumerate}

\subsubsection{La commande \ttt{tar}}
Comment mettre un répertoire entier dans un seul fichier? Il faut
utiliser la commande \ttt{tar}.

\begin{enumerate}
\item Pour mettre un répertoire dans un fichier, on fait:
  
    \com{tar cf xxx.tar répertoire}

    Exemple: \com{tar cf tesseract.tar tesseract}

    Tout le répertoire \ttt{tesseract} se retrouve stocké dans le
    fichier \ttt{tesseract.tar}. 

    \item L'opération inverse est:

\com{tar xf xxx.tar}

Exemple: \com{tar xf tesseract.tar}.
\end{enumerate}

En général, on compresse le fichier \ttt{tar} (on le crée compressé):

\begin{center}
\com{tar zcf tesseract.tar.gz tesseract}
\end{center}

et l'opération inverse est:

\begin{center}
  \com{tar zxf tesseract.tar.gz}
\end{center}

  qui extrait le répertoire contenu dans le fichier \ttt{tar}.


Donc si on récupère un fichier \ttt{.tar.gz}, on refabrique un
répertoire.

Notons que \ttt{tar} est pratique pour échanger des répertoires
entiers, par mail ou par tout autre moyen.
\subsubsection{L'alternative moderne: git}
\ttt{git} (voir \cite{git}) est outil de développeur. Il permet (entre
autres):

\begin{itemize}
\item un développement collaboratif,
\item une gestion fine des historiques, des versions successives, des
  \og forks\fg.
\end{itemize}
Écrit par Linus Thorvald pour la maintenance du noyau Linux, il est
devenu le standard de fait des développeurs \og libres\fg{} et
\emph{aussi un
moyen de distribuer les logiciels libres}. C'est un logiciel
relativement complexe pour l'utilisation comme développeur, mais
simple à utiliser  pour ce qui nous intéresse: installer des logiciels
depuis leurs sources.

Quelques entreprises, en particulier \ttt{github}\footnote{Racheté par
  Microsoft...} on développé tout un système autour de
\ttt{git}. \ttt{gitlab} propose à peu près la même chose, mais leur
logiciel est libre.

Le logiciel libre dépend \emph{très fortement} de \ttt{github}.

Parmi les ajouts de \ttt{github} (et  \ttt{gitlab}) à \ttt{git}:

\begin{itemize}
  \item L'intégration continue: à chaque \emph{push} d'un contributeur
    (envoi d'un nouveau fichier: correction ou nouveauté), on
    peut déclencher des actions: recompilation, tests, etc.
  \item Un Wiki par \og projet\fg.
  \item Des outils pour que des tiers puissent proposer des
    modifications.
  \item Des appréciations (\emph{star}).
  \item etc.
\end{itemize}

\subsubsection{Utiliser git pour installer du logiciel}

Il faut d'abord installer \ttt{git} sur la machine:
\begin{center}
\com{apt install  git}
\end{center}
  
sur les systèmes de type debian.

Ensuite, connaissant l'url du \emph{repository}, la commande (exemple):

\begin{center}
\com{git clone
  https://github.com/tesseract-ocr/tesseract.git}\footnote{\ttt{github}
et \ttt{gitlab} proposent de récupérer l'url en un clic, sur la page
web associée à chaque \emph{repository}.}
\end{center}
  
clone le répertoire (ici: \ttt{tesseract}).

Y a t-il eu des mises à jour depuis que j'ai cloné le répertoire? (la
question peut se poser un certain temps si le projet est actif). La
commande:
\begin{center}
\com{git pull}
\end{center}

 téléchargera les mises à jour. Il faut bien sûr recompiler le
code, ce qui en général se fait simplement par:

\com{make}

et:

\com{make install}\footnote{avec \ttt{sudo} ou l'équivalent devant.}

La documentation est, en principe, dans le répertoire.
