\section{Est-ce que je communique avec le bon interlocuteur?}
On sait donc chiffrer les communications. Mais il reste à résoudre le
deuxième problème, à savoir s'assurer que son interlocuteur est bien
celui voulu.
\subsection{Le cas de \ttt{ssh} (et de ses dérivés)}
\ttt{ssh} laisse à l'utilisateur le soin de vérifier que
l'interlocuteur est bien le bon, mais en l'aidant pas mal.

\begin{enumerate}
\item A la première connexion sur une machine distante\footnote{Ma commande
  est: \com{ssh dumont@math.univ-lyon1.fr}}, on reçoit ce
  genre de message:

\begin{verbatim}
The authenticity of host 'ssh-math.univ-lyon1.fr (134.214.156.14)'
can't be established.
Are you sure you want to continue connecting (yes/no)? 
\end{verbatim}

Si on répond \ttt{yes}, la connexion se poursuit avec une demande de
mot de passe, etc (rappelons que tout est crypté).

Un fichier \ttt{known\_hosts} est créé (ou rallongé) dans le
répertoire \ttt{.ssh/} de votre home directory
pour stocker non 
pas la clé publique du serveur, mais une \emph{empreinte numérique
  (hash)} de cette clé et du nom du serveur (on
va bientôt voir ce que ça veut dire).

\item Aux connections suivantes, \ttt{ssh}, en consultant le fichier
  \ttt{known\_hosts}, vérifie que rien n'a changé du coté du serveur.
  Il vous prévient d'un grave danger si quelque chose a changé:
  peut-être que vous dialoguez avec une machine qui n'est pas celle
  que vous croyez\footnote{\emph{a man in the middle?}}. Il vous
  indique quoi faire si vous voulez persister 
  dans votre volonté de vous connecter.

\end{enumerate}

On voit donc que c'est l'utilisateur qui doit s'assurer de l'identité
exacte du serveur sur lequel il se connecte. On peut quand même
ajouter que l'utilisateur doit avoir un compte et donc un mot de passe
sur le serveur 
pour pouvoir achever la connexion. Évidemment, cette technique n'est
pas adaptable à un service  web.

\subsubsection{Connexion ssh par clés (sans mot de passe)}
C'est très utile, surtout quand on veut faire des connexions
automatisées (sauvegardes par rsync, par exemple).

On est sur la machine A, et on veut se connecter par ssh sur la
machine B.

Sur A donc, on tape:

\com{ssh-keygen -t  rsa} (en répondant oui, ou rien). Cela va créer
deux fichiers dans le répertoire \ttt{.ssh/} de votre home directory
sur A:


\begin{itemize}
\item \ttt{id\_rsa}: une clé privée, à ne pas divulguer.
\item \ttt{id\_rsa.pub}: la clé publique correspondante.
\end{itemize}

Ensuite, on tape:

\com{ssh-copy-id utilisateur@B}

où ``utilisateur'' est votre identifiant sur la machine B.

Cela va recopier la clé publique \ttt{id\_rsa.pub} dans le fichier
\ttt{.ssh/authorized\_keys} de votre home directory, en tant
qu'utilisateur ``utilisateur'' sur la machine B.

Ensuite, vous n'aurez plus besoin de donner votre mot de passe!

\subsection{Empreinte numérique \textsl{(hash)}}
On aimerait, pour n'importe quel fichier, n'importe quelle suite de
caractères pouvoir créer un identifiant \emph{unique}, de taille
(nombres de signes) constante et petite.

C'est évidemment impossible: l'identifiant créé étant plus petit que
le fichier identifié, il existe forcément des fichiers différents qui
ont le même identifiant. Toutefois on peut fabriquer une méthode qui
tout en fournissant un résultat de petite taille (un petit nombre de
caractères), dépend de façon très \emph{chaotique} de ce qui lui est
donné, de sorte qu'une infime modification des données entraîne une
profonde modification du résultat.

\subsubsection{Un exemple}
On crée un fichier \ttt{c.txt} qui contient juste la chaîne de
caractères:

\ttt{Ceci est le cours Linux de l'Aldil}

J'utilise le programme \ttt{sha256} (voir \cite{sha2}) pour en
fabriquer l'empreinte: 

\begin{verbatim}
>sha256sum c.txt
6d366b28d175b579ebe8f65b40c4fbe5e3514c45b476bf1c70f72092f0b9e9e6  c.txt
\end{verbatim}
La longue chaîne de caractères produite est l'empreinte.



J'ajoute un ``.'' à la fin de la ligne:

\ttt{Ceci est le cours Linux de
  l'Aldil.}


\begin{verbatim}
>sha256sum c.txt
b8400e182df47650b59e332a8077f0852edbf1005984738dc7397d320edb8901 c.txt
\end{verbatim}

petite modification, grands effets.

\subsubsection{Quelques méthodes}
\begin{itemize}
  \item \ttt{md5sum}, considéré aujourd'hui comme trop faible pour des
    applications critiques.
  \item La famille \ttt{sha2} (\ttt{sha224sum}, \ttt{sha256sum},
    \ttt{sha384sum}, \ttt{sha512sum}): celle qu'il est recommandée
    d'utiliser. La différence entre les versions tient à la longueur
    du condensat produit et à la taille des blocs sur lesquels on
    opère.
    \item  La famille \ttt{sha3}: encore plus sûre que  les
      \ttt{sha2}, mais peu utilisée à l'heure actuelle
      (\ttt{sha384sum} est disponible sur certaines distributions).
\end{itemize}
\subsubsection{Comment ça marche?}
De manière très schématique, on regarde le matériel à hacher (fichier,
texte,...) comme une suite de bits, qu'on découpe en morceaux de
taille fixe (quitte à compléter par du remplissage). Supposons que le
hachage ait une taille de 256 bits (SHA-256). Alors, de manière
simplifié (voir \cite{sha2} pour une description plus détaillé,mais
pas forcément très claire):
\begin{enumerate}
\item On initialise ce qui sera le résultat final $\mathcal{H}$, un
  bloc de 256 bits,  avec une constante
  prédéfinies, qui fait partie de la méthode.
\item Ensuite, pour chaque morceau $\mathcal{B}$ du fichier à hacher,
  on applique un 
  certain nombre de transformations qui font interagir  $\mathcal{H}$
  et $\mathcal{B}$  pour obtenir une nouvelle version de
  $\mathcal{H}$.
  Les transformations sont:
  \begin{itemize}
  \item des ``et'' logiques (1 = vrai, 0 = faux)\footnote{vrai
    \emph{et} faux = faux, vrai  \emph{et} vrai = vrai, faux
    \emph{et} faux =faux.}.
  \item des ``ou'' logiques\footnote{vrai  \emph{ou} vrai = vrai, vrai
    \emph{ou} faux = vrai, faux \emph{ou} faux = faux.}.
  \item des rotations appliquées aux blocs de bits\footnote{Par
    exemple, pour une rotation à droite de n bits, les n bits qu'on
    pousse en dehors du bloc à droite rentrent à gauche.}.
  \item etc. A la fin, on renvoie $\mathcal{H}$.
  \end{itemize}
\end{enumerate}
C'est une idée assez voisine de \emph{mélange} (mélanger de la
peinture blanche et de la peinture noire): on veut créer un désordre
irréversible. 

    
\subsection{Garantir l'intégrité d'un message}
Bob envoie un message à Alice. Comment faire pour que Alice puisse
s'assurer que le message n'a pas été modifié?

On suppose que Bob possède une paire de clés (clé publique,
clé privée) d'un algorithme de cryptographie asymétrique et que Alice
en possède la clé publique. Alors:

\begin{enumerate}
  \item Bob calcule l'empreinte (\ttt{sha256sum} ou autre) de son
    message.
  \item Il crypte cette empreinte avec sa clé privée.
  \item Il envoie le couple (message, empreinte cryptée) à Alice.
\end{enumerate}

Alice reçoit le message et l'empreinte cryptée du message. Elle va:
\begin{enumerate}
\item Recalculer l'empreinte du message qu'elle a reçu. Appelons
  ``E1'' le résultat. 
\item Décrypter l'empreinte cryptée envoyée par Bob avec la clé
  publique de Bob, qu'elle possède. Elle obtient ``E2''.
\item Elle compare E1 et E2. S'ils sont égaux, c'est bien que:
  \begin{enumerate}
  \item le   message n'a pas été modifié.
  \item l'empreinte a bien été cryptée avec la clé privée de Bob.
  \end{enumerate}
\end{enumerate}

Mais ça ne suffit pas à tout assurer: le point faible, c'est
évidemment l'échange de clé: comment s'assurer que la clé publique que
possède Alice est bien celle de Bob? Il y a plusieurs possibilités:
\begin{enumerate}
  \item Une rencontre entre Bob et Alice (à supposer qu'ils se
    connaissent) pendant laquelle Bob donne sa clé publique à Alice.
  \item Que la clé soit stockée chez un tiers de confiance.
\end{enumerate}
\subsection{Certificats et tiers de confiance}
Les tiers de confiance s'appellent \emph{Autorité de
  Certification}. Ils délivrent des \emph{certificats}.

\subsubsection{Certificats} Ce sont des fichiers à la structure
normalisée (norme X-509). Un certificat
contient principalement:
\begin{itemize}
\item Un numéro de version.
\item Un numéro de série.
\item L'algorithme de signature du certificat.
\item DN (Distinguished Name)\footnote{Des informations du genre \og
  état civil\fg{}: Nom, adressse, etc.}.
\item Validité (dates limites:  pas avant,  pas après).
\item DN de l'objet du certificat.
\item Informations sur la clé publique:
  \begin{itemize}
  \item     L'algorithme de la clé publique.
  \item     La clé publique proprement dite.
  \end{itemize}
\item Et des champs optionnels.
\end{itemize}

\textbf{et} la signature des informations ci-dessus par l'autorité de
certification: une empreinte numérique  des informations qui est
\textbf{crypté} avec la clé \textbf{privée de l'autorité de certification.}

Ensuite:

\begin{enumerate}
  \item Le certificat est conservé par le serveur (ce n'est pas
  forcément un serveur web; d'autres services peuvent utiliser des
  certificats, mais on peut supposer que c'est un serveur web, pour
  simplifier). 
  \item Le certificat est présenté au client que se connecte au
    serveur (vous, avec votre navigateur).
  \item Le client décrypte le condensat avec la clé publique de
    l'autorité de certification et récupère la clé publique du serveur.
  \item Si tout va bien, il contacte l'autorité de certification pour
    vérifier que le certificat n'a pas été révoqué.
\end{enumerate}
 \textbf{Mais} d'où le client (vous) connaît il la clé publique du
 serveur qui lui présente son certificat? \og \emph{Les navigateurs web
 modernes intègrent nativement une liste de certificats provenant de
 différentes Autorités de Certification choisies selon des règles
 internes définies par les développeurs du navigateur} \fg{}: il peut
 donc y avoir un fonctionnement en cascade, une autorité de
 certification étant certifiée par une autre. Et bien sûr il y a un
 moment où il faut bien faire confiance.

 La révocation par une autorité est possible, par exemple en cas de
  compromission. Remarquons aussi que les certificats ont une
 durée de vie limitée.

 Quelques  autorités de certification:

 \begin{itemize}
 \item La liste \url{https://webgate.ec.europa.eu/tl-browser/#/tl/FR}
   montre des autorités françaises qui sont soit publiques (Caisse des dépôts
   et consignations, Ministère de l'intérieur etc.),  soit privées,
   certaines assiociées à des activités particulières (notaires,
   expert-comptables etc.) d'autre vendant simplement leurs services.
   
 \item \ttt{Lets'Encrypt} est une autorité de certification a but non
   lucratif: \url{https://letsencrypt.org/fr/}, financée, entre
   autres, par des fournisseurs de services et hébergeurs comme OVH en
   France.
 \end{itemize}
   
