\section{Le shell: configuration, environnement}
\label{initshell}
\begin{enumerate}
\item Dans le home directory, quelques fichiers cachés\footnote{Les
  fichiers et les répertoires cachés ont un nom qui commence par ``.''.
  La commande \com{ls} ne les montre pas, il faut ajouter l'option
  \ttt{a}: \com{ls -a} pour les voir.} sont importants.

  
  \dx{
    \com{ls \~ /.*bash*}

 ~ }{
%\begin{verbatim}
/home/moi/.bash\_history  /home/moi/.bashrc
/home/moi/.bash\_logout 
%\end{verbatim}

~
}{(chez moi)}

Et il existe aussi, toujours dans le home directory, un fichier de nom
\ttt{.profile}

\item
  Dans \ttt{/etc}

  \dx{
    \com{ls /etc/bash*} }
  {
%\begin{verbatim}
/etc/bash.bashrc  /etc/bash\_completion
/etc/bash\_completion.d:

\textsf{apport\_completion}

\textsf{git-prompt  gmic}
%\end{verbatim}
  }{(chez moi)}
  
  et

  \dx{\com{ls /etc/profi*}}
{
/etc/profile

/etc/profile.d:

\textsf{01-locale-fix.sh gawk.sh}

\textsf{vte-2.91.sh}

\textsf{bash\_completion.sh}


}{(chez moi --abrégé-- )}


\end{enumerate}

\`A quoi tout cela sert-il?
\begin{itemize}

\item \`A définir des \emph{variables d'environnement.}
\item \`A définir ou à modifier des commandes.
\item Ce qui a un nom contenant \ttt{completion} sert à definir les
  règles de complétion automatique (quand on tape un TAB).
\end{itemize}

Les variables d'environnement sont globales et peuvent être utilisées
par les programmes pour leur permettre d'adapter leur
comportement. Un exemple typique: la langue utilisée.

\begin{itemize}
  \item les fichiers \ttt{profile} sont indépendants du shell utilisé.
  \item L'ordre de parcours est :
    \begin{itemize}
    \item d'abord le(s) fichiers \ttt{profile} de \ttt{/etc},
    \item puis \ttt{/etc/bash.bashrc},
    \item puis le fichier \ttt{\~/.profile}
    \item puis \ttt{\~/.bashrc}.
    \end{itemize}
\end{itemize}

On peut donc, dans les fichiers  \ttt{\~/.profile} et \ttt{\~/.bashrc}
compléter ce qui est fait dans \ttt{/etc} ou redéfinir des variables.

\subsection*{Qu'est-ce qui est défini?}

La commande

\com{printenv}

permet de voir tout ça. Il y a beaucoup de variables définies!

Remarquons entre autres:

\begin{itemize}
\item \ttt{SHELL=/bin/bash}

  Ici, la variable est \ttt{SHELL} et elle contient
  \ttt{/bin/bash}. C'est le shell qu'on utilise bien sûr.
\item Des variables dont le nom commence par \ttt{LC}: définissent la
  langue utilisée et les jeux de caractères utilisés.
\item etc.
\end{itemize}

On affecte des valeurs dans le shell en faisant par exemple:

\com{x=''coucou''}

mais pour voir ce que contient la variable, il faut faire:

\com{echo \$x}

\subsection*{Une variable importante: le \ttt{path}}\label{path}
Essayez:

\com{echo \$PATH}

ou

\com{printenv|grep PATH}

On obtient une liste de chemins dans le système de fichiers, ou les
chemins sont séparés par \ttt{:}. Exemple (chez moi):

  
\noindent\ttt{/home/moi/.local/bin:/usr/local/sbin:/usr/local/bin:}\newline \ttt{/usr/sbin:/usr/bin:/sbin:/bin:/usr/games:/usr/local/games:/snap/bin 
}

\begin{itemize}
\item Quand on tape une commande, le shell va la chercher
  successivement dans chacun des chemins de \ttt{PATH}, dans l'ordre,
  et exécute la première trouvée.

\item On peut modifier le \ttt{PATH}. Pour rajouter un chemin, on peut
  faire (ici je rajoute \ttt{/opt/}):

  \com{PATH=/opt/:\$PATH} (ajout au début)

  ou:

  \com{PATH=\$PATH:/opt/} (ajout à la fin).

  \com{echo \$PATH} pour voir le résultat.

  C'est typiquement le genre de commande qu'on peut rajouter dans
  \ttt{\~/.bashrc}. 
\end{itemize}
