\section{Une liste de commandes Unix}
\begin{tabular}{|l|l|l|l|}
  \hline
 
  Commande& Objet& Exemple &Options importantes, remarques\\
  \hline
   \hline
  diff & Différences entre fichiers&\com{diff a b}& utiliser entre
  fichiers texte\\
  \hline
  file& Devine le type de fichier& \com{file *.pdf}&\\
  \hline
  mv& Déplace, renomme (voir plus bas)&&\\
  \hline
  sort&Trier un fichier &\com{sort fich}& ajouter \ttt{-n} pour un tri
  numérique\\
  \hline
  touch&modifie la date ou crée un fichier vide&\com{touch toto}&\\
  \hline
  chmod&change les droits (voir plus bas)&&\\
  \hline
  which& trouver une commande dans le PATH&\com{which uname}&\\
  \hline
  whoami& votre login&&\\
  \hline
  uptime&temps d'activité (depuis le reboot)&&\\
  \hline
  date&date et heure&&plein d'options (man date)\\
  \hline
  df& espace disque utilisé et libre&&\com{df -h}, sortie \og
  humaine\fg\\
  \hline
\end{tabular}\medskip

\paragraph{Quelques détails:}
\begin{itemize}
\item \ttt{mv}: déplacer ou renommer. C'est assimilé à une écriture,
  et donc il faut avoir le droit d'\emph{écrire} sur l'objet qui peut
  être un fichier ou un répertoire.

  Exemples:
  \begin{itemize}
    \item \com{mv toto /tmp} déplace le fichier ou le répertoire
      \ttt{toto} dans \ttt{/tmp} (il faut aussi avoir le droit
      d'écrire dans le répetoire but (ici, avec \ttt{/tmp}, c'est le cas).
    \item \com{mv toto /tmp/tutu}, on déplace \ttt{toto} dans
      \ttt{/tmp} où il s'appelera \ttt{tutu}.
    \item \com{mv toto machin}. Le déplacement se réduit à un
      changement de nom.
  \end{itemize}
\item \ttt{chmod}: change les droits d'un fichier ou d'un
  répertoire. Il y a plusieurs façons de procéder, une assez facile,
  l'autre moins!
  \begin{itemize}
    \item Pour changer les droits du propriétaire, l'option est
      \ttt{u}, c'est \ttt{g} pour celle du groupe et \ttt{o} pour
      le reste du monde. Quelques exemples:
      \begin{enumerate}
      \item \com{chmod u+w toto},
      \item \com{chmod u+x g+x o-w toto},
      \item On peut \og factoriser\fg{} les ordres:
        \com{chmod ug+x o-w toto} par exemple.
      \item La méthode \og dure\fg{}: on considère les 9 digits
        possibles \ttt{xxxyyyzzz}; on met un 1 quand on veut que le
        droit soit ouvert, 0 sinon; ainsi, pour obtenir
        \ttt{rw-r-----}, ceci correspond à: \ttt{110 100 000}. Bien
        maintenant, il s'agit de 3  nombres binaires; il faut les calculer
        en base 10:
        \begin{enumerate}
        \item \ttt{110}= $1\times 4 + 1 \times 2 +0 =6$,
        \item \ttt{100}= $1\times 4 + 0 \times 2 +0 = 4$
        \item \ttt{000}= $0$.
        \end{enumerate}
        
        Et la commande est \com{chmod 640 toto}.

        Amusant, non?
        (excellent exercice de calcul mental).
      \end{enumerate}
  \end{itemize}
\end{itemize}
