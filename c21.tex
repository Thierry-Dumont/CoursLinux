
\centerline{
  \begin{minipage}{0.65\textwidth}
    \textsl{Dans la suite la syntaxe:}

    \begin{center}
      \com{commande arguments}
    \end{center}

    \textsl{désignera
une commande (à taper dans le shell); le \og prompt\fg{}, c'est à dire
le début de la ligne de commande peut varier d'une configuration à l'autre;
} \com{} \textsl{représente ce prompt et n'est pas à taper.}
  \end{minipage}
}
\section{Utilisateurs et groupes}
\subsection{Les utilisateurs}

Les utilisateurs sont définis dans le fichier
\texttt{/etc/passwd}

Regardons\footnote{\com{less /etc/passwd} (par
  exemple).\label{refnote1}}  une ligne du ficher, correspondant à la description de
l'utilisateur de nom \ttt{moi}:  
\begin{center}
\texttt{moi:x:1000:1000:moi,,,:/home/moi:/bin/bash}
\end{center}

On y voit des champs séparés par \texttt{\og : \fg} qui sont:
\begin{enumerate}
\item Le nom de connexion de l'utilisateur (\og login\fg) (\texttt{moi}, ici).
\item \texttt{x}: qui  contenait jadis le mot de passe.
\item \texttt{1000}: en pratique, c'est un nombre qui identifie
  l'utilisateur, et pas son nom (plus rapide, plus simple). Ici,
  \texttt{1000} et \texttt{moi} sont associés et identifient le même
  utilisateur. \texttt{1000} est l'\texttt{uid} de \texttt{moi}. Et
  l'\texttt{uid}  est évidemment unique, des utilisateurs différents
  ont des \texttt{uid} différents.
\item \texttt{1000}: identifiant du groupe de l'utilisateur.
  \item \texttt{moi,,,}: le \og vrai \fg{}nom (j'aurais pu mettre
   \texttt{Thierry Dumont}), suivi de commentaires optionnels.
  \item Le home directory (\texttt{/home/moi}).
  \item Le chemin vers le shell utilisé, ici c'est \texttt{bash}.
  \end{enumerate}

Les utilisateurs appartiennent à des groupes (un ou plusieurs). Dans le
fichier \ttt{/etc/passwd} on décrit le groupe \emph{principal} auquel
l'utilisateur appartient. \emph{Dans beaucoup de distributions Linux,
  le groupe principal a le même nom que l'utilisateur (moi, ici) et ne
  contient qu'un seul utilisateur! Mais rien n'empêche de regrouper
  les utilisateurs dans différents groupes, ce qui, on va le voir
  permettrait de définir des ensembles d'utilisateurs plus ou moins
  privilégies par exemple.}\smallskip

\rmq{
  \begin{itemize}
    \item En parcourant le fichier \texttt{passwd}\footref{refnote1},
      on remarque que pour certains utilisateurs, le shell est 
      \texttt{/usr/sbin/nologin} ou \texttt{false}: on ne peut pas se
      connecter au système sous ses noms d'utilisateur, mais on en verra
      l'intérêt quand on regardera les \emph{processus}.
    \item Commandes:
      
      \com{adduser} permet d'ajouter un utilisateur,
      
        \com{deluser} d'en supprimer un.
      
        \noindent Vous ne pourrez pas utiliser
      ces commandes sous votre login habituel: l'explication viendra
      par la suite.
  \end{itemize}
  }

\paragraph{Un utilisateur pas comme les autres: \textsf{root}.}
\begin{center}
  \texttt{root:x:0:0:root:/root:/bin/bash}
\end{center}
\textsf{root} a tous les droits! même de faire

\com{rm -f /}

Il
convient donc \emph{d'être très prudent} quand on prend cette identité, ce qui
est absolument nécessaire pour accéder à certaines parties du système,
configurer, installer des logiciels etc. On regardera ça par la suite.

\subsection{Les groupes (d'utilisateurs)}
Ils sont décrits dans le fichier \ttt{/etc/group} qui pour chaque groupe
donne:
\begin{itemize}
\item Le nom du groupe.
\item x : (champ désactivé).
\item Le group-id (gid), un nombre unique.
\item La liste des utilisateurs qui appartiennent à ce groupe.
\end{itemize}

Par exemple, su ma machine, le fichier \ttt{/etc/group} contient la ligne:

\texttt{lpadmin:x:121:moi}.

Ici, l'utilisateur ``moi''
appartient au groupe lpadmin. On peut conjecturer que ``lpadmin'' a à
voir avec les imprimantes (\textsl{line printer admininistration}) et
qu'être membre de ce groupe va donner des 
droits d'administration particuliers.



\exo{}
\`A quels groupes est ce que j'appartiens en tant qu'utilisateur?

(utiliser \com{grep} et \com{/etc/group}\footnote{Ou plus simplement
  la commande \com{groups} (sans argument).}).
\section{Droits}
\subsection{Commençons par un petit essai}
Dans le fichier \ttt{/etc/passwd}, le champ du mot de passe est
remplacé par un \ttt{x}, le mot de passe effectif étant dans
\ttt{/etc/shadow}.

\exo{Essayez de regarder le fichier /etc/shadow
  (\com{less /etc/shadow}). Que se passe-t-il?}\smallskip

Il y a visiblement des problèmes de permission (droits): c'est normal,
car on a affaire à un fichier \textsl{sensible},
mais comment est-ce que ça marche? Il faut s'intéresser aux \emph{droits}.
\subsection{Droits concernant les fichiers, les répertoires etc.}~


Résultat de:

\com{ls -l /etc/passwd}

\begin{center}
\large\texttt{-rw-r---r--- 1 root root 3444 déc. 6 11:42 /etc/passwd}
\end{center}

Le premier tiret de


\begin{center}
\texttt{-rw-r---r--- 1 root root 3444
  déc. 6 11:42 /etc/passwd}
\end{center}

indique qu'on a affaire à un fichier;
  à la place du \textbf{-} on aurait \textbf{d} pour un
  répertoire.


Ensuite, il y a 3 triplets de 3 caractères (ici \texttt{rw-},
\texttt{r---} et \texttt{r---}).
Dans chaque triplet le
premier caractère peut être \ttt{r} ou \ttt{-}, le second  \ttt{w} ou \ttt{-} et
le troisième \ttt{x} ou \ttt{-}.

\begin{itemize}
  \item \ttt{r} indique le droit de lire, \ttt{-} l'impossibilité de
    lire.
  \item \ttt{w} indique le droit d'écrire, \ttt{-} l'impossibilité
    d'écrire. Attention, écrire doit être pris au sens le plus large:
    modifier, déplacer dans l'arborescence, renommer ou effacer, tout
    ce qui peut modifier.
  \item \ttt{x} indique:
    \begin{itemize}
      \item pour un fichier, le droit d'exécuter, c'est à dire que le
        fichier est une  commande, un programme exécutable.
      \item pour un répertoire, c'est le droit de traverser le
        répertoire, ce qui n'implique pas de lire son contenu!
    \end{itemize}
    Et \ttt{-} indique l'absence de ce droit.
\end{itemize}

Que signifient chacun de ces 3 triplets?

\begin{enumerate}
  \item Le premier, les droits de l'utilisateur (root dans notre
    cas). Qui, ici, peut 
    donc lire le fichier et écrire dessus (modifier, effacer,
    déplacer, etc!!!).
  \item Le second, les droits des autres membres de son groupe (en
    pratique, dans ce cas précis, il n'y en a sûrement pas): seulement
    lire dans le   cas présent.
  \item Le troisième: les droits du reste du monde, c'est à dire de
    tous les autres utilisateurs: ici, c'est seulement le droit de lire.
\end{enumerate}

Peut-on changer les droits d'un fichier ou d'un répertoire? Oui, à
condition qu'il vous appartienne, dans ce cas la commande est
\com{chmod}; on va en voir un exemple plus loin.
\subsection{Autres exemples}
\begin{itemize}
\item Pour \texttt{/etc/shadow}, les droits sont:

  \begin{center}
  \texttt{-rw-r--------- 1 root shadow 1763 déc. 6 11:42
      /etc/shadow}
  \end{center}
  
  \noindent et donc le fichier appartient à root et au groupe shadow. Vu les
  droits, on voit qu'un utilisateur  qui n'est pas
  ``root'' et qui n'appartient pas à ``shadow'' ne peut pas lire ce
  fichier.
  Notons que si on ajoutait un utilisateur  comme ``moi'' au
  groupe ``shadow'', alors il pourrait lire ce fichier.
  
  \item \texttt{bash}: le shell. Pour savoir où il est, la commande
    est:

    \com{which bash}

    Le résultat est probablement
    \texttt{/usr/bin/bash}. Ensuite 

    \com{ls -l /usr/bin/bash}

    donne:

    \texttt{-rwxr-xr-x 1 root root 1183448 juin  18  2020
        /usr/bin/bash}

    Remarquez les ``x'' pour tout le monde: c'est normal, car il
    s'agit d'un programme
    utilisé par tout le monde.
 

\end{itemize}

\exo{Étudiez la racine du système de fichiers en tapant: \com{ls -l/}}
 
\exo{}
  \begin{enumerate}
  \item \com{cd /tmp} (pour être dans un coin tranquille).
  \item Créez des répertoires emboîtés:

    \com{mkdir toto}

    \com{mkdir toto/tutu}\footnote{On peut faire ça en une seule
      commande \com{mkdir -p toto/tutu}}

    Créez un fichier vide dans tutu de nom ``file'' (ou ce que vous
    voulez):

    \com{touch toto/tutu/file} 
  \item Pour voir les droits affectés à toto: \com{ls -dl toto}. Cela
    devrait donner (en remplaçant moi par votre nom d'utilisateur):
    
    \texttt{drwxrwxr-x 3 moi moi 4096 déc.  24 15:23 toto}
  \item Vérifiez que les commandes:

    \com{ls toto} et \com{ls toto/tutu} fonctionnent.
  \item Maintenant changez les droits de toto:

    \com{chmod chmod ugo-rw toto}

    (ce qui veut dire ``enlever les
    droits de lire et d'écrire à l'utilisateur (u), au groupe (g) et
    aux autres (o)).

    \com{ls -dl toto} vous indique les nouveaux droits:

    \texttt{d---x---x---x 3 moi moi 4096 déc.  24 15:23 toto}

    On ne peut plus lire ni écrire.
  \item Regardez maintenant ce que donnent les commandes:
    \begin{itemize}
    \item     \com{cd toto} puis \com{ls toto}.
    \item Maintenant que vous êtes dans \ttt{toto}, tapez  \com{cd
      tutu} ou \com{cd /tmp/toto/tutu}. Vérifiez que vous pouvez
      écrire dans tutu
    (\com{touch xx} par exemple).
    \end{itemize}
  \item Retour dans \ttt{/tmp}: \com{cd /tmp}. Essayez d'effacer
    \ttt{toto}. Comme 
    il n'est pas vide, la commande \com{rmdir} ne peut pas
    fonctionner. Il faut utiliser \com{rm -r toto}. Que se passe-t-il?
    
  \item Redonnons nous le droit d'écrire sur et dans toto:
    \com{chmod u+w toto}.
    Essayons à nouveau \com{rm -r toto}.

    Pourquoi est ce que ça
    ne marche pas? Parce que \com{rm -r} a besoin de lire le contenu
    de \ttt{toto} pour l'effacer. On tape \com{chmod u+r toto} et là, la
      commande \com{rm -r toto} efface tout. Subtil!
    
    
\end{enumerate}


\section{Processus}
Dans un guide Unix/Linux on peut lire: \emph{\og Un processus est une
instance d'un programme en train de s'exécuter, une tâche. Le shell
crée un nouveau processus pour exécuter chaque
commande\fg}. Le terme \emph{instance} faisant partie du sabir
informatique (le mot n'est pas français), on va tenter d'illustrer ça:\smallskip

\com{which firefox}

Réponse:

\ttt{/usr/bin/firefox}.

Le programme firefox est donc stocké dans le fichier \ttt{/usr/bin/firefox}.

Que se passe-t-il si je tape \com{firefox} en ligne de
commande? Le shell (bash) va chercher\footnote{dans une liste
  de chemins bien définis -- on verra ça plus tard --} un fichier de nom
firefox\footnote{il y a forcément plein de \texttt{x} dans les droits
  du fichier \ttt{/usr/bin/firefox}: voir plus
  haut!} et, après l'avoir trouvé,
et de manière un peu simplifiée, demander au noyau linux de le copier
en mémoire et de lancer son exécution.

\`A l'exécution, ce n'est pas seulement une copie qui réside en
mémoire, mais aussi des données\footnote{Pensez par exemple à
  libreoffice: les données, c'est le texte que vous écrivez.},  une
description des fichiers ouverts 
et un \og état programme\fg, qui pointe vers l'instruction en cours
dans le programme.

\`A partir de ce moment là,
cette ``copie'' (= instance) devient pendant toute la durée de son
exécution un \textsf{processus}.

 Le processus ne peut pas accéder en dehors des 
zones de mémoire que le système lui a alloué.


\`A chaque processus sont associés:
\begin{enumerate}
\item un numéro (PID, \textsl{process id}) unique et chaque
processus appartient à un utilisateur et hérite des droits (de lire,
d'écrire,...) de l'utilisateur \emph{et} des groupes auxquels il
appartient. D'autre part, un utilisateur n'a aucun droit sur les
processus des autres utilisateurs.
\item un processus \emph{père}, repéré par son PPID (\textsl{parent
  process identifier}): \emph{tout processus} a un père, à une
  exception près, car 
  il faut bien  un ancêtre commun: c'est le
  processus \ttt{init} qui a le PID 1. Tous les autres en descendent
  et ont des PID $>$ 1.
\end{enumerate}
Pour pouvoir faire quelques expériences, il faut d'abord apprendre à
maîtriser l'outil qui permet de  voir quels sont les processus
présents et leurs caractéristiques:  la
commande \com{ps} suivie d'options.

\subsection{La commande \com{ps}}
La commande \com{ps} est un peu bizarre.
\begin{itemize}
\item Si vous tapez juste \com{ps}, vous ne verrez que les processus
  qui dépendent de votre shell (terminal) soit en général juste
  \ttt{bash} et \ttt{ps} (puisque justement, vous êtes en train de
  faire tourner...\ttt{ps}):
\begin{verbatim}
    PID TTY          TIME CMD
 338023 pts/5    00:00:00 bash
 371762 pts/5    00:00:00 ps
\end{verbatim}
mais vous voyez quand même les \ttt{PID}, le terminal \ttt{TTY} auquel
ils sont attachés, le temps calcul dépensé \ttt{TIME} et le nom de la
commande \ttt{CMD}.
\item Si vous tapez \com{ps -e}, vous voyez tous les processus en
  cours. Je recommande de taper

  \com{ps -e | less}

  car il y a beaucoup
  de processus en cours! Mais ça ne dit quand même pas grand chose
  sur chaque processus.
\item \com{ps -F -U moi} (remplacez \com{moi} par votre nom
  d'utilisateur): vous montre tous \emph{vos} processus en vous disant
  beaucoup de choses.

  \com{ps -F -U moi|less}

  ou

  \com{ps -F -U  moi|head}\footnote{\texttt{head} montre tes premières
    lignes d'un fichier. De même, il ya \texttt{tail}...}
  
  vont vous permettre de voir le début de cette liste. La
  première ligne est:
\begin{verbatim}
  UID   PID    PPID  C    SZ   RSS PSR STIME TTY   TIME CMD
\end{verbatim}
Détaillons un peu:
\begin{itemize}
\item  \ttt{UID PID PPID}: on a vu ça plus haut.
\item  \ttt{C}: pourcentage d'utilisation du processeur. 
\item  \ttt{SZ  RSS}: la taille mémoire réservée par le processus et la
  taille réellement utilisée.
\item   \ttt{STIME}: date de démarrage du processus
\item   \ttt{TTY}: le terminal depuis lequel la commande a été
  lancée. Mais si la commande n'a pas été lancée depuis un terminal,
  on a un ``?''.
\item \ttt{TIME}: le temps UC consommé par le processus depuis son démarrage.
\item \ttt{CMD}: la commande.
\end{itemize}
\item Avec

  \com{ps -aef}

  on voit de manière étendue les
  processus de tous les utilisateurs.
\end{itemize}

\exo{
Commencez par agrandir au maximum votre fenêtre de ligne de commande
(en général la touche F1 le fait (F1 aussi pour revenir à un terminal
de taille normale)).

Ensuite:

\com{ps -aef | less}

Vous devriez voir quelque chose qui ressemble à ça:

\begin{verbatim}
UID          PID    PPID  C STIME TTY          TIME CMD
root           1       0  0 déc.13 ?      00:00:12 /lib/systemd/systemd --system --deserialize 16
root           2       0  0 déc.13 ?      00:00:00 [kthreadd]
root           3       2  0 déc.13 ?      00:00:00 [rcu_gp]
root           4       2  0 déc.13 ?      00:00:00 [rcu_par_gp]
root           6       2  0 déc.13 ?      00:00:00 [kworker/0:0H-kblockd]
root           9       2  0 déc.13 ?      00:00:00 [mm_percpu_wq]
root          10       2  0 déc.13 ?      00:00:07 [ksoftirqd/0]
root          11       2  0 déc.13 ?      00:07:10 [rcu_sched]
root          12       2  0 déc.13 ?      00:00:02 [migration/0]
\end{verbatim}

Ce qui s'analyse ainsi: 
\begin{itemize}
\item Le process 1 est \ttt{systemd} qui va lancer... beaucoup de
  choses. Sur d'autres distribution (la mienne est une Ubuntu 20/10)
  cela peut être \ttt{init} (c'est très \og Linux canal
  historique\fg).
\item Si on regarde les processus suivants, on voit qu'ils ont des PID
  croissants, mais qu'ils sont lancés (PPID) par le processus de PID =
  2.
\item Supposons que firefox est lancé (sinon, il n'y a qu'à le
  lancer!).
  
  \com{ps -aef |grep firefox}

  On obtient à peu près (la sortie est tronquée à droite\footnote{et chez
    vous, ça va forcément être un peu différent.}):
\begin{verbatim}
moi       373641  337755  6 16:58 ?        00:00:13 /usr/lib/firefox/firefox
moi       373755  373641  2 16:58 ?        00:00:06 /usr/lib/firefox/firefox ...
moi       373801  373641  0 16:58 ?        00:00:00 /usr/lib/firefox/firefox ..
moi       373956  373641  1 16:59 ?        00:00:02 /usr/lib/firefox/firefox ...
moi       374054  373641  3 17:00 ?        00:00:04 /usr/lib/firefox/firefox ...
moi       374100  373641  2 17:00 ?        00:00:03 /usr/lib/firefox/firefox ...
\end{verbatim}
Ce qui s'analyse ainsi (faire la même analyse chez vous):
\begin{enumerate}
\item Il y a 6 processus. les deuxième et les suivants ont un PPID =
  373641, et donc ils ont été lancés par le processus de PID = 373641,
  qui est le premier de la liste\footnote{Pourquoi firefox lance-t-il autant
  de processus? probablement pour des questions de performances et/ou
   de sécurité.}.
\item Mais si on regarde la première ligne, on voit aussi que le
  processus de PID = 373641 au PPID = 337755. Qui donc a lancé ce
  ``premier'' firefox?

  \com{ps -aef |grep 337755}

 et vous aurez son ``père''\footnote{chez moi, c'est un processus de xfce4, car
 j'utilise xfce, pas gnome.}.
\end{enumerate}
\end{itemize}

  }
  
\subsection{D'autres commandes associées aux processus}
\subsubsection{Voir les processus tourner:}
La commande \com{top} vous montre les processus actifs en tête. On
peut jouer sur 
le taux de rafraîchissement, et tuer un processus (\ttt{k}).

\subsubsection{Arrêter un processus: la commande \com{kill}}
Cette commande permet d'envoyer un signal à un processus (en pratique,
on tue le processus). La syntaxe est: \ttt{kill} suivi d'au moins un
espace et le PID du process à tuer\footnote{Pour les processus
  récalcitrants, on peut faire \com{kill -9} PID.}.

\exo{
Chercher à nouveau les processus  firefox comme ci-dessus, et tuer le
premier. Chez moi cela donne:

\com{kill 373641}

En procédant ainsi tue le père... et ses descendants $=>$ plus aucun  firefox ne
tourne.
On peut aussi tuer un des ses fils.

  }
\subsection{Autres  notions sur les processus}
\begin{enumerate}
\item Espace utilisateur et espace noyau: la mémoire est coupée en
  deux parties, l'espace utilisateur et l'espace noyau. Tous vos
  processus tournent dans l'espace utilisateur, et ceux du noyau dans
  l'espace noyau. Vous ne pouvez pas intervenir sur les processus qui
  tournent dans l'espace noyau: c'est une sécurité
  supplémentaire. Avec les CPUs Intel, ces protections sont mêmes
  assurées par le processeurs (les \emph{rings}).
\item Un processus peut créer des processus fils. On l'a déjà vu avec
  \ttt{initd}. Une notion un peu différente est le \emph{fork} ou un
  processus crée une copie de lui même.

\item Les processus peuvent communiquer entre eux. Il existe pour cela
  différents mécanismes (exemples: une zone de mémoire partagée, les
  sockets (une sorte de réseau interne à la machine), le \emph{pipe}
  qu'on a vu avec la commande \texttt{|} etc.
  
\item Un processus peut créer des \emph{processus légers} appelés
  \emph{threads}. La différence avec la création de processus classiques,
  c'est principalement que tous les \emph{threads} et leur père
  partagent la même zone mémoire. Si la commande \com{top} vous montre
  un programme qui consomme plus de 100\% d'UC, c'est que plusieurs
  threads tournent en parallèle. Vous pouvez installer le package
  \ttt{htop} qui permet de bien visualiser ça. L'intérêt des
  \emph{processus légers} est la rapidité de la communication entre
  eux puisqu'il n'y a aucun mécanisme de partage de données à mettre
  en jeu\footnote{C'est en général là-dessus que repose la parallélisme
    en mémoire partagée: si vous utilisez des logiciels de vidéo par
    exemple, ils 
    sont en général multithreadés, tant le genre de calcul qu'ils font
    est coûteux. Avec $n$ c{\oe}urs, on peut en théorie calculer $n$ fois
    plus vite qu'avec un seul.}. Le défaut, c'est que leur
  programmation est délicate (partage de mémoire $=>$ pas de
  protection d'accès mémoire entre threads).

\item \textbf{Les démons (daemons):}
 Ce   sont des programmes résidents qui sont à priori  chargés au
 démarrage.
 Quelques exemples: \ttt{cupsd}: gestionnaire d'impression,
 \ttt{rsyslogd}: gestion des ``logs'', \ttt{ntpd}: synchronisation de
 l'horloge avec une horloge distante, etc.
\end{enumerate}


