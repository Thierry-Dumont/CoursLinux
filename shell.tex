\section{Le shell: configuration, environnement}
\begin{enumerate}
\item Dans le home directory, quelques fichiers cachés\footnote{Les
  fichiers et les répertoires cachés ont un nom qui commence par ``.''.
  La commande \com{ls} ne les montre pas, il faut ajouter l'option
  \ttt{a}: \com{ls -a} pour les voir.} sont importants.

  
  \dx{
    \com{ls ~/.*bash*}

 ~ }{
%\begin{verbatim}
/home/moi/.bash\_history  /home/moi/.bashrc
/home/moi/.bash\_logout 
%\end{verbatim}

~
}{(chez moi)}

Et il existe aussi, toujours dans le home directory, un fichier de nom
\ttt{.profile}

\item
  Dans \ttt{/etc}

  \dx{
    \com{ls /etc/bash*} }
  {
%\begin{verbatim}
/etc/bash.bashrc  /etc/bash\_completion
/etc/bash\_completion.d:

apport\_completion

git-prompt  gmic
%\end{verbatim}
  }{(chez moi)}
  
  et

  \dx{\com{ls /etc/profi*}}
{
/etc/profile

/etc/profile.d:

01-locale-fix.sh gawk.sh

vte-2.91.sh

bash\_completion.sh


}{(chez moi --abrégé-- )}


\end{enumerate}

\`A quoi tout cela sert-il?
\begin{itemize}

\item \`A définir des \emph{variables d'environement.}
\item \`A définir ou à modifier des commandes.
\end{itemize}

Les variables d'environement sont globales et peucvent être utilisées
par les programmes pour leur permettre d'adapter leur comportement.
