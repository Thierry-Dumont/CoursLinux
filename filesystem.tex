\section{Systèmes de fichiers}
Définition approximative: une organisation hiérarchiques de fichiers
et de répertoires.

C'est une définition à peu près indépendante du support matériel de
ces systèmes (voir l'article de Wikipédia \cite{fs}).

En pratique sous Unix, à chaque fichier ou répertoire est associé un
\textsf{inode}, qui contient tous les renseignements le
concernant. Chaque inode a un numéro, unique. La commande:

\com{ls -i objet}

permet d'obtenir le numéro d'inode de \ttt{objet} (ce qui ne sert pas
  à grand chose).

  Mais les  \sffs, s'ils montrent au système le même type de
  structure (arborescence de fichiers et répertoires), ne stockent pas 
  tous en interne les données de la même façon (une bibliothèque est
  associée à chaque type de  \sff), d'où des fonctionalités et des
  performances différentes.
  Citons, pour Linux:

  \begin{itemize}
    \item La famille \ttt{ext}: \ttt{ext1}, \ttt{ext2}, \ttt{ext3} et
      \ttt{ext4}.
      Les premières versions (1 et 2) ont été réalisées par Remy
      Card, intervenant fréquent aux premières Journées du Logiciel
      Libre à Lyon  \cite{rcard}.
      C'est le  \sff{} standard sous Linux.

    \item \ttt{Jfs}: (IBM; peu gourmand en ressources).
    \item \ttt{ReiserFS}: semble abandonné.
    \item \ttt{Xfs}: très grande capacité de stockage, et rapide pour
      les très grands  \sffs.
    \item \ttt{Btrfs}: selon les auteurs, c'est le \emph{B-Tree file-system}
      ou le \emph{Better file system}. Génial sur le papier, mais ne
      semble pas percer pour l'instant (standard cependant chez Suse).
    \item \ttt{Zfs}: (Open-Zfs). Origine SUN, assez proche de
      \ttt{Btrfs} qui semble en dériver, en tout cas conceptuellement.

    \item Les systèmes Linux peuvent aussi manipuler (et créer) des
      \sffs{} propriétaires (\ttt{ntfs}, \ttt{vfat}, \ttt{fat} etc.). 

    \item Systèmes de fichiers  en réseau: une machine \emph{exporte} une
     partie de son arborescence vers d'autres machines; sous Unix (et
     donc Linux), cela s'appelle \ttt{nfs} (network file
     system); \ttt{samba} fait ça aussi. 
   \item  Il existe aussi des \emph{clustered file-systems} pour la lecture
     et l'écriture en parallèle (Lustre, BeeGFS, GPFS, Xtremefs,
     GlusterFS, MooseFS, XrootD, EOS...); ça c'est pour Youtube ou le
     Centre de Calcul des Physiciens Nucléaires à La Doua\footnote{Ou
       pour le futur télescope à grand champ Vera-C.~Rubin: 20
       téraoctets par nuit, à conserver au moins 10 ans \cite{verarubin}.}.
  \end{itemize}

  \subsection*{Quelques détails:}
  \begin{itemize}
    \item \ttt{Btrfs} et \ttt{Zfs} permettent de faire des
      \emph{snapshots}, soit en français des \emph{instantanés}:
      obtenir une \emph{vue} du  \sff{} tel qu'il est  à
     un instant donné, en le
      laissant évoluer par ailleurs. On
      trouvera en annexe une description (à coup sûr simplifiée) de la
      technique utilisée (le COW).
      \item \ttt{Btrfs} et  \ttt{ReiserFS} utilisent tous les deux une
        représentation des données sous forme d'\emph{arbre binaire
          équilibré} (\emph{B-Tree}), décrite aussi en annexe, qui assure une
        très grande vitesse de lecture.
  \end{itemize}

  Un des problèmes majeurs est la fiabilité des systèmes de fichiers:
  outre qu'un bug pourrait avoir des conséquences désastreuses, il
  faut absolument tenir compte de l'accident qu'un arrêt brutal de la
  machine (coupure de courant par exemple) pourrait causer. Le risque
  est d'avoir un
  système de fichiers incohérent, car l'interruption aura eu lieu
  alors qu'une écriture était en cours (apparition d'orphelins  par
  exemple). Regardons pour cela l'histoire des systèmes 
  \ttt{ext[1-4]}:
  \begin{itemize}
    \item \ttt{ext1} était un prototype; \ttt{ext2} (Rémy Card, 1993
      \cite{rcard}) 
      est rapidement devenu le système standard sous Linux. Il limite
      la fragmentation 
      du support, n'implante pas de sécurité et a fini par apparaître
      comme limité quant à la taille et au nombre des objets stockés.
    \item \ttt{ext3} (2001): le principal apport est la \emph{journalisation}
      qui permet de \emph{rejouer} une E/S qui aurait été interrompue.
    \item \ttt{ext4} (2008): la limite de la taille est portée à
      $2^{60}$ (!) octets. Une autre avantage est de réduire encore la
      fragmentation. C'est actuellement le \sff{} sous la
      quasi totalité des distributions Linux. 
  \end{itemize}
  
\paragraph{La journalisation:} On consultera \cite{journ} pour une
description détaillée.
%% Un système de fichiers comprend des
%% méta-données (où les données sont-elles stockées? comment?, le nom,
%% les droits etc.) et des données. La journalisation logique (par défaut
%% dans \ttt{ext3}) consiste à enregistrer les \emph{modifications} apportées
%% aux méta données dans le \emph{journal} après avoir écrit les
%% données.
Le principe de la \emph{journalisation} consiste à maintenir une liste
(en général dans un fichier) des modifications apportées au système de
fichiers, de manière indépendante des modifications elles-mêmes, pour
pouvoir rejouer ces modifications en cas de crash, ou au moins de
remettre le système dans un état cohérent. C'est ce qui est implanté
dans \ttt{ext3} (avec différentes options, voir référence ci-dessus).  

\subsection{Manipulation des systèmes de fichiers (partitionner,créer, monter,
  etc.)}
Tout, ici, doit être fait en tant que \ttt{root}, donc 
\textdbend \textbf{Attention, danger!}

On ne va parler que des systèmes de fichiers classiques, physiques
(\ttt{ext4}, \ttt{vfat} etc. Pour \ttt{btrfs}, voir par
exemple\cite{btrfs}).

\begin{enumerate}
\item Un \sff{} s'installe toujours dans une \emph{partition.} Même si
  le périphérique sur lequel vous allez installer le \sff{} ne va
  en contenir qu'un, vous devez partitionner.
\item On crée ensuite le \sff{} en utilisant la commande \ttt{mkfs}
  qui installe dans la partition la structure de donnée nécessaire,
  prête à recevoir vos fichiers et répertoires.
\item Ensuite il faut \emph{monter} le \sff{} pour qu'il soit
  incorporé à l'arborescence de la machine (ceci vaut aussi pour les
  \sff{} en réseau).
\end{enumerate}
\subsubsection{Créer une (des) partitions(s), formater}
En ligne de commande, la commande est \ttt{fdisk}. Un outil graphique,
plus simple à utiliser, est \ttt{gparted}.

La première chose à faire est de connaître le nom du \emph{device}
(périphérique à formater). Répétons: il faut être \ttt{root}, donc:
\begin{itemize}
\item soit vous êtes connecté en tant que  \ttt{root} (par \com{su -}
  par exemple),
\item soit vous préfacez toutes les commandes ci-dessous par \ttt{sudo}
  si votre machine est amie de sudo. Vous pouvez aussi faire une fois
  pour toutes, dans ce cas: \com{sudo -i}.
\end{itemize}
Dans tous les cas: \textdbend!

Pour connaître le périphérique que vous voulez formater vous pouvez
par exemple, dans le cas d'un périphérique usb, regarder ce que donne
la commande: 

\com{dmesg}

après avoir introduit le périphérique. Vous verrez par exemple quelque
chose comme :

\begin{center}
\texttt{[sdd] Attached SCSI removable disk}.
\end{center}


Vous pouvez aussi installer et utiliser la commande \ttt{lsscsi}, bien
pratique.
Il faut ensuite taper une commande comme \ttt{df} pour voir si le
périphérique est monté. Si oui, le démonter (voir plus bas).

On lance la commande

\com{fdisk /dev/sdd} (si votre périphérique est bien
\ttt{/dev/sdd}). Ensuite, la commande est interactive; 
il faut:
\begin{enumerate}
  \item Effacer les partitions existantes (\ttt{p} pour les lister,
    \ttt{d} pour en effacer une).
  \item \ttt{n} pour créer une partition (il y en a différents types,
    mais utilisons celle par défaut pour linux). On crée une partition primaire,
    et on en donne la taille (par défaut on utilise tout le périphérique).
   \item On peut comme ça créer plusieurs partitions, qui vont
     s'appeler \ttt{/dev/sdd1}, \ttt{/dev/sdd2} etc (parce que mon
     périphérique est \ttt{/dev/sdd}).
   \item \ttt{w} pour écrire les résultats; En fait jusque là, on n'a
     rien changé sur le disque. En sautant cette étape, on peut encore
     tout sauver!
   \item \ttt{q} pour quitter.
\end{enumerate}
\subsubsection{Installer un  \sff{} (formatter)}
C'est très simple. Mes partitions s'appellent par exemple
\ttt{/dev/sdd1}, \ttt{/dev/sdd2} etc. Pour installer un \sff{}
  \ttt{ext4}, il suffit de faire:

  \com{mkfs -t ext4 /dev/sdd1} (et bien sûr remplacer 1 par ce qu'il
  faut le cas échéant...).

\subsubsection{Monter le \sff{}}


Monter un \sff{} consiste à \emph{accrocher} un \sff{} à un
répertoire existant 
dans l'arborescence. Je peux utiliser un répertoire \emph{existant}
(attention, je vais cacher ce qui est éventuellement dedans), ou en
créer un. Supposons que j'utilise \ttt{/mnt}. La commande est alors:

\com{mount -t ext4 /dev/sdd1 /mnt}

Il y a 3 arguments:
\begin{enumerate}
\item le type de \sff, ici: \ttt{-t ext4},
\item ce qu'on monte, ici: \ttt{/dev/sdd1},
\item où on le monte, ici:  \ttt{/mnt}.
\end{enumerate}
Et voilà.

Dans notre cas, \ttt{/mnt} est accessible. Ensuite, rien ne prouve que
le répertoire ainsi monté sera accessible à tout le monde: à priori, il
va appartenir à \ttt{root}. Il faut donc en changer les droits, ou le
propriétaire. On va
voir une autre façon de faire.

\subsubsection{Démonter le \sff}
Il faut utiliser \ttt{umount}. Dans l'exemple ci-dessus on peut soit
faire:

\com{umount /mnt}

soit:

\com{umount /dev/sdd1}

\subsection{Le fichier \ttt{/etc/fstab}}
Si on souhaite automatiser le \og mount \fg{} de \sffs, il faut
utiliser le fichier \ttt{/etc/fstab}, et donc le modifier\footnote{
  Il faut bien sûr avoir les pouvoirs de \ttt{root}. Ensuite, toucher
  à ce fichier est dangereux. Il faut:
  \begin{enumerate}
  \item Le copier pour en garder la version actuelle: \com{cp
    /etc/fstab /etc/fstab.save}

    C'est une recommandation qui vaut pour tous les fichiers \og sensibles\fg.
  \item Utiliser un éditeur de texte pour le modifier. Vous en avez
    plusieurs à disposition: \ttt{vim} (pour les masochistes),
    \ttt{gedit} ou \ttt{nano}, voir même \ttt{emacs}. Pour ce genre de
    chose, je recommande \ttt{nano}.
  \end{enumerate}
  }.

\ttt{/etc/fstab} contient la description de tout ce qui doit ou peut
être monté, en précisant quoi doit être monté et où.

\subsubsection{\ttt{fstab}, à l'ancienne}

Exemple:


{\ttfamily
\begin{tabular}{llllll}
\# <file system>& <mount point> &  <type>&  <options>&       <dump>&  <pass>\\
/dev/sda2 & / &ext4 & errors=remount-ro &0  & 1\\
/dev/sda1 & /boot/efi & vfat&  umask=0077 & 0 &  1\\
\end{tabular}
}

Le premier champ est ce qu'on monte, le deuxième où on le monte et le
troisième le type de \sff{}. On voit ainsi que la partition
\ttt{/dev/sda1}, formatée en \ttt{vfat} sert pour le boot (c'est
  uefi). Passons sur le champ ``dump'', le dernier champ dit l'ordre
  dans lequel les \sffs{} sont vérifiés au boot.
  Les options sont nombreuses, ici:
  \begin{itemize}
    \item \ttt{errors=remount-ro} dit de remonter sans droits
      d'écriture en cas d'erreur,
    \item \ttt{ umask=0077} indique les droits qu'on enlève: ainsi 077
      laisse tous les droits \ttt{rwx} au propriétaire (root), et
      enlève tous les droits (car 7 en binaire est 111) au groupe et
      aux ``autres''. Les droits de \ttt{/boot/efi} sont donc
      \ttt{rwx-----------}.
  \end{itemize}

  \subsubsection{\ttt{fstab}, plus moderne}
  Citer les noms des partitions comme \ttt{/dev/sda1} est source
  d'erreur. Le système propose un identificateur unique pour chaque
  partition, qu'on peut obtenir avec la commande \ttt{blkid}. Chez
  moi, cela donne:

  \com{blkid /dev/sda1}

  \ttt{/dev/sda1: UUID="4454-89BE" BLOCK\_SIZE="512" TYPE="vfat"
    PARTLABEL="EFI System Partition"
    PARTUUID="ad32b00b-6810-45f6-93d2-d946eb611524"}\smallskip

  \com{blkid /dev/sda2}

  \ttt{/dev/sda2: UUID="c447c9ae-2491-4526-8fd8-8a56f79b2e0b"
    BLOCK\_SIZE="4096" TYPE="ext4"
    PARTUUID="2eb752b3-d027-4383-b811-48f999adf3a2"}\medskip

  on récupère les \ttt{UID=...} et le fichier \ttt{/etc/fstab}
  devient:\smallskip
  
{\ttfamily
\begin{tabular}{llllll}
\# <file system>& <mount point> &  <type>&  <options>&       <dump>&  <pass>\\
 UUID="c447c9ae-2491-4526-8fd8-8a56f79b2e0b"& / &ext4 & errors=remount-ro &0  & 1\\
UUID="4454-89BE"& /boot/efi & vfat&  umask=0077 & 0 &  1\\
\end{tabular}    
}
\subsubsection{Autoriser un utilisateur à monter un \sff}
Dans notre cas (monter \ttt{/dev/sdd1} sur \ttt{/mnt}), la ligne a
ajouter à \ttt{/etc/fstab} est: 

{\ttfamily
\begin{tabular}{llllll}
  /dev/sdd1 & /mnt & ext4 & noauto,user \\
\end{tabular}
}
Ici:
\begin{itemize}
  \item \ttt{user} donne le droit de monter le \sff{} aux
    utilisateurs (et donc aussi de démonter).
  \item \ttt{noauto}: le \sff{} ne sera pas monté au moment du boot.
\end{itemize}

On peut aussi  récupérer l'UUID et écrire quelque chose comme:

{\ttfamily
\begin{tabular}{llllll}
UUID="6e9d5173-3e30-4e8b-8546-b82a7a311c57" & /mnt & ext4 &
noauto,user \\
\end{tabular}
}\medskip

N'importe quel utilisateur peut alors taper:

\com{mount /mnt}

et

\com{umount /mnt}
